\documentclass{article}
\usepackage[utf8]{inputenc}

\usepackage{amsmath, amssymb, amsthm, thmtools, thm-restate, tikz, graphicx,
  hyperref, cleveref, comment, expl3, xparse, ebproof, enumitem, xcolor,
  stmaryrd, backnaur, verbatim}

\usepackage[titlenumbered, ruled]{algorithm2e}
\usetikzlibrary{shapes.geometric}
\usetikzlibrary{positioning}
\usetikzlibrary{cd}

\hypersetup{
  linktoc=all,
  allbordercolors={0 0 0},
  pdfborderstyle={/S/U/W 1}
}

\usepackage[mathscr]{euscript}

% https://tex.stackexchange.com/questions/37797/theorem-environment-line-break-after-label
\newtheoremstyle{break}% name
{}%         Space above, empty = `usual value'
{}%         Space below
{\itshape}% Body font
{}%         Indent amount (empty = no indent, \parindent = para indent)
{\bfseries}% Thm head font
{.}%        Punctuation after thm head
{\newline}% Space after thm head: \newline = linebreak
{}%         Thm head spec

% https://tex.stackexchange.com/questions/250035/transform-output-theoremstyleremark-from-italics-to-bold
\newtheoremstyle{boldremark}
{\dimexpr\topsep/2\relax} % space above
{\dimexpr\topsep/2\relax} % space below
{}          % body font
{}          % indent amount
{\itshape\bfseries}% theorem head font
{.}         % punctuation after theorem head
{\newline}      % space after theorem head
{}          % theorem hed spec. (empty = "normal")

\newtheoremstyle{discussion}% name
{}%         Space above, empty = `usual value'
{}%         Space below
{}% Body font
{}%         Indent amount (empty = no indent, \parindent = para indent)
{\bfseries}% Thm head font
{.}%        Punctuation after thm head
{\newline}% Space after thm head: \newline = linebreak
{}%         Thm head spec

\theoremstyle{definition}
\newtheorem{definition}{Definition}[section]
\newtheorem{notation}{Notation}[section]

\theoremstyle{break}
\newtheorem{theorem}{Theorem}[section]
\crefname{theorem}{Theorem}{Theorems}

\theoremstyle{break}
\newtheorem{corollary}{Corollary}[theorem]
\crefname{corollary}{Corollary}{Corollaries}

\theoremstyle{break}
\newtheorem{lemma}[theorem]{Lemma}
\crefname{lemma}{Lemma}{Lemmas}

\theoremstyle{remark}
\newtheorem*{remark}{Remark}

% https://latex.org/forum/viewtopic.php?t=24225
\theoremstyle{discussion}
\newtheorem{discussion}[theorem]{Discussion}
\crefname{discussion}{Discussion}{Discussions}

\theoremstyle{boldremark}
\newtheorem*{boldremark}{Remark}

% \newcommand{\powerset}{\mathscr{P}}
\newcommand{\powerset}{\mathbb{P}}
\newcommand{\N}{\mathbb{N}}
\newcommand{\Z}{\mathbb{Z}}
\newcommand{\R}{\mathbb{R}}

\newcommand{\fact}{\mathbf{fact}}

% Partial function arrow
% https://math.stackexchange.com/questions/742614/notation-for-partial-function-set
\newcommand{\pto}{\multimap\mspace{-2mu} \to}

\DeclareMathOperator{\Type}{Type}
\DeclareMathOperator{\FV}{FV}
\DeclareMathOperator{\Pii}{Pi}
\DeclareMathOperator{\foralll}{forall}
\DeclareMathOperator{\fun}{fun}
\DeclareMathOperator{\deff}{def}
\DeclareMathOperator{\axiom}{axiom}
\DeclareMathOperator{\checkk}{check}
\DeclareMathOperator{\evall}{eval}
\DeclareMathOperator{\wf}{wf}
\DeclareMathOperator{\fst}{fst}
\DeclareMathOperator{\snd}{snd}

% \DeclareMathOperator{\lett}{let}
% \DeclareMathOperator{\inn}{in}

% https://math.stackexchange.com/questions/298648/is-there-a-common-symbol-for-concatenating-two-finite-sequences
\newcommand\mdoubleplus{\mathbin{+\mkern-10mu+}}


\begin{document}

\section{Introduction}
Our language is a theorem prover for constructive logic, based on 
the Calculus of Constructions (CoC), which is a dependently typed lambda calculus.
CoC forms the foundation of many theorem provers like Coq, Lean and Agda, 
though they also include many other features like inductive types and a 
cumulative hierarchy of type universes.

In this document, we provide a big-step semantics for our language, upon which
our interpreter was based.

Our approach to semantics is heavily inspired by 
\href{https://homepages.inf.ed.ac.uk/gdp/publications/sos_jlap.pdf}
{Plotkin's seminal work} on structural operational semantics, as well as the 
\href{https://smlfamily.github.io/sml97-defn.pdf}{specification for Standard ML}.
Though our language specification is formal and mathematically
intensive, we believe in the importance of rigour to ensure precision and clarity.
This is especially important for our language since dependent type theory is a
complex subject.

A benefit of this approach is that not only did our specification serve as a
guide for our implementation, many of the rules we formalize here
translate almost directly into the code which we implemented.

Despite our formality, we still aim to provide sufficient exposition and
justification for our design decisions.
We hope that the reader will agree that the ideas underlying our formalization
are intuitive. 

In the next section, we offer some background on dependent type theory in 
relation to the Curry-Howard correspondence. Thereafter, we discuss where our
language stands in relation to CoC.

\subsection{Dependent type theory and Curry-Howard}
Abstractly, CoC is a pure type system that sits at the top of Barendregt's
lambda cube.
More concretely, it generalizes the simply typed and polymorphic
lambda calculi by replacing the simple function type with the dependent Pi type.
These generalize the simple function type $A \to B$ by allowing the output type
$B$ to now \textit{depend on the value} of the input expression.
The type of functions is now written as $\Pi_{x : A} B(x)$ where
we write $B(x)$ for the return type to emphasize that $x$ may appear free in
$B$.

The set theoretic analogue to this dependent Pi type is the generalized
cartesian product. Given a set $A$ and a family of sets $\langle B_x \, | \, x
\in A\rangle$ indexed by the
elements $x \in A$, we can form the generalized cartesian product, denoted
\begin{align*}
  \Pi_{x \in A} B_x = \bigg\{ f : A \rightarrow \bigcup_{x \in A} B_x \, | \, \forall x \in A, \, f(x) \in B_x \bigg\}
\end{align*}
Functions that inhabit this set are known as \textit{choice functions} in set
theory. Such choice functions associate to each $x \in A$, an element $f(x)$ in $B_x$.
% As a fun fact, the Axiom of Choice asserts that this set is nonempty
% if every $B_x$ is inhabited.

% The main references which we base our language on are
% \href{https://www.andres-loeh.de/LambdaPi/LambdaPi.pdf}{this paper on $\lambda
%   \Pi$} and \href{https://github.com/andrejbauer/spartan-type-theory}{this
%   implementation of a dependent type theory}.

More generally, dependently typed languages allow types to depend on expressions
as well.

Now you might ask, of what use are these dependent types?
Remember that the Curry-Howard correspondence says that simple type theory
can be used to encode constructive propositional logic,
with sum types (ie coproducts) corresponding to disjunction, products to 
conjunction, unit to $\top$ and void to $\bot$.

But what about the quantifiers $\forall$ and $\exists$?
This is where dependent type theory comes in.

To prove the universally quantified statement, $\forall x \in A, \, P(x)$ by hand,
we would assume that $x$ is an arbitrary element of $A$ and then proceed to prove
that $P(x)$ holds.
Well, according to the BHK interpretation of constructive logic, constructing
a proof as such is akin to defining a function which takes 
as input an element $x$ of type $A$ and returns a proof of $P(x)$.
What is the type of such a function? It is precisely the Pi type, 
$\Pi_{x : A} P(x)$!

In the event that the output type of a Pi type doesn't depend on the value of the
input, we recover the simple function type, which we write as $A \rightarrow B$
or $\Pi_{x : A} B$. So we see that the Pi type allows us to model both
logical implication as well as universal quantification. 

As for the existential quantifier, a proof of $\exists x \in A, \, P(x)$
corresponds to a pair, $\langle x, \, \text{pf} \rangle$ where $x$ is an element
of type $A$ witnessing the existential and pf is a proof that for this $x$, 
$P(x)$ holds. The type of the pair $\langle x, \, \text{pf}\rangle$ is the
Sigma type, $\Sigma_{x : A} P(x)$. Such Sigma types generalize the ordinary pair
type, by allowing the type of the second component to depend on the value of
the first. 

There is also another interesting interpretation of the Sigma type.
Consider the notion of a subset, $\{x \in A \, | \, P(x) \}$, from set theory.
By interpreting sets as types, we can model elements of this subtype of $A$ as
pairs containing an element of type $A$ and a second element certifying that the
first element does belong to this subtype. What would such a certificate be?
Well, a proof that it satisfies the predicate $P$!
Such a pair is of the Sigma type $\Sigma_{x : A} P(x)$!

Note that if the second component of a pair doesn't depend on the first, we recover
the usual pair (ie product) type, written $A \times B$ or $\Sigma_{x : A} B$.
With this, we see that the Sigma type can encode 3 different notions, namely
conjunction, existential quantification, and the subtype/subset relation!

In essence, adding dependent types to the lambda calculus allows types to include
expressions like variables.
This give us the extra tools that we need to go beyond propositional logic and
formalize constructive predicate logic!

The table below summarizes the correspondence between types and their Curry-Howard
interpretation.

\begin{center}
\begin{tabular}{ |c|c|c|c| }
  \hline
 Type & Formal notation & Alt notation & Curry-Howard interpretation \\ 
 \hline
 Pi & $\Pi_{x : A}B(x)$ & & $\forall x \in A, \, B(x)$ \\  
    & $\Pi_{x : A}B$ & $A \rightarrow B$ & $A \rightarrow B$ \\  
    \hline
 Sigma & $\Sigma_{x : A}B(x)$ & & $\exists x \in A, \, B(x)$ \\
       & $\Sigma_{x : A}B(x)$ & & $\{ x \in A \, | \, B(x) \}$ \\
       & $\Sigma_{x : A}B$ & $A \times B$ & $A \wedge B$ \\
       \hline
 Sum & $A + B$ & & $A \vee B$ \\
 \hline
 Unit & $\top$ & & $\top$ (ie logical truth) \\
 \hline
 Void & $\bot$ & & $\bot$ (ie logical falsity) \\
 \hline
\end{tabular}
\end{center}

These are the types which we implement in our language.
In the next section, we discuss how our language resembles and differs from
CoC.

\subsection{Our language in comparison to CoC}
As mentioned earlier, our language is based on the original
formulation of CoC. This means that we neither have inductive types nor a
countable hierarchy of type universes.
Since our emphasis here is on simplicity, we stay true to the original calculus,
by offering only 2 sorts, \verb|Type| and \verb|Kind|.

As with CoC, \verb|Type| in our language is the sort of propositions as well as 
datatypes. Informally, we can think of type as the type of all types.
In other words, an element $T$ of type \verb|Type| can be interpreted as either
being a proposition whose elements are proofs or a datatype like a set.
Though this is in line with the Curry-Howard interpretation, this has the
unfortunate consequence of conflating the syntactic roles of propositions and
data types.
  
This may be counter-intuitive for users because in first (or higher) order
logic, there is a clear distinction between terms and well-formed formulae,
which belong to different syntactic categories. 

As for the sort \verb|Kind|, it's actually not really useful for users. It's
just here to prevent Girard's paradox, a type theoretic formulation of the set
theoretic Burali-Forti argument.
Through the construction of Girard's paradox, one can (in theory) provide a 
proof of false, ie a lambda expression that inhabits the empty type.
In doing so, the logical consistency of the calculus is lost.

It should also be noted that we focus on theorem proving via Curry-Howard over
general programming. This means that our language, like CoC, does not have
basic programming constructs like general recursion, conditional statements or
even basic types like numbers and booleans.

In its original formulation as a pure type system, CoC only includes one built-in
type, namely the Pi type. It does not have Sigmas or sums (ie coproducts).
While there are ways to encode $\exists$, $\wedge$ and $\vee$ into CoC
[insert ref to Type Theory and Formal Proof book], such encodings can be awkward
to work with. Hence, we decided to also implement Sigma and sum types as built-ins
so that $\exists$ and $\wedge$ can be modeled more naturally via Sigma and
$\vee$ via the sum type.

% Unfortunately, it should be noted that our formalization of the typing
% judgments for the Sigma type does allow for Girard's paradox to creep back to our
% language, when viewed as a logical calculus. We discuss this issue in more detail
% and offer possible workarounds in a later section, along with the other
% metatheoretic properties of our language.

% Another core feature of our language will be type inference.
% This will be implemented alongside typechecking, using a fancy
% \textit{bidirectional typechecking} algorithm. Don't worry, we'll formalize all
% this later. For now, this just means that typechecking and type inference are
% mutually recursive processes.

% Our language also doesn't have full blown recursion and is
% \textit{strongly normalizing} in that every sequence of beta reductions will
% always terminate in a unique head normal form. This allows us to safely
% normalize all terms to full head normal form.

\section{Syntax}
In order to implement our rich type system, we promote types to the level of
expressions in our language, so we no longer have separate syntactic categories 
for them.

Yes, this means that \textit{everything} in our language is an expression, even
those things which we call types!
With this in mind, a type is then an expression, say $T$, which satisfies a very
specific typing judgment, ie $\Gamma \vdash T \Leftarrow s$. We'll revisit
this again later.

For now, we describe the concrete ascii syntax for our language, alongside
the syntactic sugar users can enter. This includes unicode symbols.

At the top level, programs will be nonempty sequences of statements.
Statements function like top-level commands, as found in other theorem provers
like Coq and Lean. These offer a way for users to interact with our system.
Through statements, users can introduce global assumptions and definitions.

\subsection{Syntax for expressions}
We first define the set of valid expressions of our language as the initial algebra, 
ie least prefixed point, of the endofunctor
\begin{align*}
  F : X \mapsto \{ E \, | \, \{ E_0, E_1, \dots, E_n \} \subseteq X \text{ and }
    \begin{prooftree}   
      \hypo{E_0}
      \hypo{E_1}
      \hypo{\dots}
      \hypo{E_n}
      \infer4{E}
    \end{prooftree}
    \text{ is a rule instance}
  \} 
\end{align*}
on some complete powerset lattice.
Such a construction can be justified by the famous fixed point theorems of Tarski
and Kleene.

Before we list the full set of rules, we first introduce some metavariables.

\subsubsection{Metavariables}
\begin{enumerate}
  \item $A$, $B$, $T$, $E$ range over expressions. 
  \item $x$, $y$, $z$ range over variables.
\end{enumerate}

\subsubsection{Syntax rules}
\begin{enumerate}
\item \textbf{Sorts} \\
  \[
    \begin{prooftree}
      \infer0{\Type}
    \end{prooftree}
  \]

  \[
    \begin{prooftree}
      \infer0{\Kind}
    \end{prooftree}
  \]

  Since our language makes no distinction between types and propositions,
  we allow users to enter \verb|Prop| (short for ``proposition") in place of 
  \verb|Type|.

\item \textbf{Variables} \\
  \[
    \begin{prooftree}
      \infer0{x}
    \end{prooftree}
  \]

\item \textbf{Optional parentheses} \\
  \[
    \begin{prooftree}
      \hypo{E}
      \infer1{( \, E \, )}
    \end{prooftree}
  \]
  
\item \textbf{Function abstraction} \\
  \[
    \begin{prooftree}
      \hypo{x}
      \hypo{E}
      \infer2{\fun \, x => E}
    \end{prooftree}
  \]
  \[
    \begin{prooftree}
      \hypo{x_0}
      \hypo{x_1}
      \hypo{\dots}
      \hypo{x_n}
      \hypo{E}
      \infer5{\fun \, x_0 \,\, x_1 \, \dots \, x_n => E}
    \end{prooftree}
  \]
  Note that all functions in our language are unary and so
  $\fun \, x_0 \, x_1 \, \dots \, x_n => E$ is syntactic sugar for
  \[ \fun \, x_0 => (\fun \, x_1 => \dots (\fun x_n => E)) \]

  Simimarly, one can also provide optional type annotations for input variables.
  This is to help the type checker infer the type of a function.
  \[
    \begin{prooftree}
      \hypo{x}
      \hypo{T}
      \hypo{E}
      \infer3{\fun \, (x : T) \, => E}
    \end{prooftree}
  \]

  \[
    \begin{prooftree}
      \hypo{x_i}
      \hypo{T_i}
      \hypo{E}
      \infer3{\fun \, (x_0 : T_0) \,\, (x_1 : T_1) \, \dots \, (x_n : T_n) => E}
    \end{prooftree}
  \]
  We also treat $\fun \, (x_0 : T_0) \,\, (x_1 : T_1) \, \dots \, (x_n : T_n) => E$
  as syntactic sugar for
  \[ \fun \, (x_0 : T_0) => (\fun \, (x_1 : T_1) => \dots (\fun (x_n : T_n) => E)) \]

  As syntactic sugar, we allow users to use \verb|lambda| and $\lambda$ in place
  of \verb|fun|. 

\item \textbf{Pi and Sigma type} \\
  \[
    \begin{prooftree}
      \hypo{x}
      \hypo{A}
      \hypo{B}
      \infer3{\Pii \, (x : A), \, B}
    \end{prooftree}
  \] 

  \[
    \begin{prooftree}
      \hypo{x_i}
      \hypo{A_i}
      \hypo{B}
      \infer3{\Pii \, (x_0 : A_0) \, (x_1 : A_1) \, \dots \, (x_n : A_n), \, B}
    \end{prooftree}
  \] 
  As with functions, this is syntactic sugar for
  \[ \Pii \, (x_0 : A_0), \, (\Pii \, (x_1 : A_1), \,  \dots \, (\Pii \, (x_n : A_n), \, B)) \]

  If there is only one pair of input type and type annotation following the \verb|Pi|,
  the brackets may be ommitted so users can enter \verb|Pi x : A, B| instead of
  \verb|Pi (x : A), B|.

  The syntax rules for Sigma, ie $\Sigmaa\, (x : A), B$ and the above syntactic
  sugar work the same as with Pi above. 

  We also allow users to enter, in place of \verb|Pi|, \verb|forall| or using
  unicode, $\forall$ and $\Pi$.
  Similarly, users can enter \verb|exists|, $\Sigma$ and $\exists$ instead of
  \verb|Sigma|.

  In the event that the output type does not depend on the input type, users
  may enter \verb|A -> B| and \verb|A * B| in place of 
  \verb|Pi (x : A), B| and \verb|Sigma (x : A), B|.

  \item \textbf{Sigma constructor} \\
  \[
    \begin{prooftree}
      \hypo{E_1}
      \hypo{E_2}
      \infer2{(E_1, \, E_2)}
    \end{prooftree}  
  \]

  \item \textbf{Sigma eliminators} \\
  \[
    \begin{prooftree}
      \hypo{E}
      \infer1{\fst \, E}
    \end{prooftree}
  \]

  \[
   \begin{prooftree}
    \hypo{E}
    \infer1{\snd \, E}
   \end{prooftree}
 \]

\item \textbf{Type ascriptions} \\
  \[
    \begin{prooftree}
      \hypo{E}
      \hypo{T}
      \infer2{(E : T)}
    \end{prooftree}
  \]
  This functions similarly to other functional languages in that it's used
  mainly to provide type annotations. For instance, it can be used to help the
  type checker if it's unable to infer the type of an expression.

\item \textbf{Local let bindings} \\
\[
  \begin{prooftree}
    \hypo{x}
    \hypo{E}
    \hypo{E'}
    \infer3{\text{let } x := E \text{ in } E'}
  \end{prooftree}
\]

\item \textbf{Sum type} \\
\[
  \begin{prooftree}
    \hypo{A}
    \hypo{B}
    \infer2{A + B}
  \end{prooftree}
\]

\item \textbf{Sum type constructor} \\
\[
  \begin{prooftree}
    \hypo{E}
    \infer1{\inl E}
  \end{prooftree}
\]

\[
  \begin{prooftree}
    \hypo{E}
    \infer1{\inr E}
  \end{prooftree}
\]

These are meant for introducing the left and right components of a disjoint sum,
ie $\inl E$ constructs an expression of type $A + B$ given an expression $E$ of
type $A$. Similarly, $\inr E$ constructs an $A + B$ given $E$ of type $B$.

\item \textbf{Sum type eliminator} \\
Given expressions $E$ and variables $x$ and $y$, users can perform case analysis
on a sum type via the \verb|match| construct below

\begin{verbatim}
  match E with
  | inl x -> E1
  | inr y -> E2
  end
\end{verbatim}

The ordering of both clauses can be swapped so users can also enter
\begin{verbatim}
  match E with
  | inr y -> E2
  | inl x -> E1
  end
\end{verbatim}

Note here that $x$ is bound in the expression that is $E_1$, while $y$ is bound
in $E_2$.

This \verb|match| construct is fashioned after the similarly named pattern
matching construct in ML like languages.
However, unlike those, we do not implement actual pattern matching since pattern
matching for dependent types is not an easy problem.
Our \verb|match| construct serves the sole purpose of allowing one to perform
case analysis on sums.

For convenience, we write 
$\match(E, \, x \rightarrow E_1, \, y \rightarrow E_2)$ to denote this
construct.

\end{enumerate}

\subsubsection{Syntactic sugar: wildcard variables}
In the event where the user does not intend to use the variable being bound,
like say in a lambda expression, the underscore, \verb|_|, can be used in place
of a variable name.
For instance, one can enter \verb|fun _ => Type| in place of \verb|fun x => Type|.
As in other languages, \verb|_| is not a valid identifier that can be used
anywhere else in an expression. It can only be used in place of a variable in a
binder.

\subsubsection{A word about notation}
Note that we often write $\lambda x, \, E$ instead of $\fun \, x => E$ as in the
concrete syntax. Similarly, we also write $\lambda (x : T), \, E$ in place of
$\fun \, (x : T) => x$.
Finally, we use $\Pi_{x : A}B(x)$ to abbreviate $\Pii \, (x : A), \, B$.

\subsection{Syntax for statements}
Statements are top level commands through which users interact with our
language. Programs in our language will be \textit{nonempty} sequences of
these statements.

\begin{enumerate}
\item \textbf{Def} \\
\[
  \begin{prooftree}
    \hypo{x}
    \hypo{E}
    \infer2{\deff \, x := E}
  \end{prooftree}
\]
This creates a top level, global definition, binding the variable $x$ to the
expression given by $E$.

\item \textbf{Axiom} \\
\[
  \begin{prooftree}
    \hypo{x}
    \hypo{T}
    \infer2{\axiom \, x : T}
  \end{prooftree}
\]
This defines the variable $x$ to have a type of $T$ with an unknown binding.
The interpreter will treat $x$ like an unknown, indeterminate constant.

We also allow users to enter \verb|constant x : T| instead of \verb|axiom x : T|,
as syntactic sugar. Note that since our system conflates the notion of types and
propositions, introducing a constant \verb|x| of type \verb|T| is akin to
introducing an assumption (ie axiom).

\item \textbf{Check} \\
\[
  \begin{prooftree}
    \hypo{E}
    \infer1{\checkk \,\, E}
  \end{prooftree}
\]
This instructs the interpreter to compute the type of the expression $E$ and
output it.

\item \textbf{Eval} \\
\[
  \begin{prooftree}
    \hypo{E}
    \infer1{\evall \,\, E}
  \end{prooftree}
\]
This instructs the interpreter to fully normalize the expression $E$.

Later, we will properly define the notion of normalization using a
big-step semantics. For now it suffices to say that normalizing an expression
is the process of fully evaluating it until no more simplifications can be
performed.

\end{enumerate}

\section{Capture avoiding substitutions}
In this section, we define the concept of free variables and capture avoiding
substitutions. This will be important when we later define normalization and
reduction in our big-step semantics.

\subsection{Free variables}
We define the set of free variables of an expression $E$, ie $\FV(E)$ recursively.
\begin{align*}
  \FV (x) &:= \{x\} \\
  \FV (E_1 \, E_2) &:= \FV(E_1) \cup \FV(E_2) \\
  \FV (\lambda \, x, \, E(x)) &:= \FV(E) \setminus \{x\} \\
  \FV (\lambda \, (x : T), \, E(x)) &:= \FV(T) \cup (\FV(E) \setminus \{x\}) \\
  \FV (\text{let } x := E \text{ in } E') &:= \FV((\lambda x, \, E') \,\, E) \\
  \FV (\Pi_{x : A} B(x)) &:= \FV(A) \cup (\FV(B) \setminus \{x\})
\end{align*}
Note that in Pi expressions, the input type $A$ is actually an expression itself
and so may contain free variables. It's important to note that the $\Pi_{x :
  A}$, like a lambda is a binder binding $x$ in $B$. However, this does not bind
$x$ in the input type, $A$. If $x$ does appear in the input type $A$, it is
free there.

\begin{align*}
 \FV (\Sigma_{x : A} B(x) &:= \FV(\Pi_{x : A} B(x))
\end{align*}

Sigma expressions follow the same rules as with Pi expressions, with $x$ being
bound in the output type $B$ but not in the input type $A$.

\begin{align*}
  \FV (A + B) &:= \FV(A) \cup \FV(B) \\
  \FV (\match(E, x \rightarrow E_1, y \rightarrow E_2)) &:= 
    E \cup (E_1 \setminus \{x\}) \cup (E_2 \setminus \{y\}))
\end{align*}
Note that in match expressions, $x$ is bound in $E_1$ and $y$ is bound in $E_2$.

For the expressions $\fst \, E$, $\snd \, E$, $\inl \, E$ and $\inr \, E$, the
set of free variables is precisely $\FV(E)$, since those keywords are treated
like constants.

\subsection{Substitution}
Here we define $E [x \mapsto E'] := E''$ to mean substituting all free
occurrences of $x$ by $E'$ in the expression $E$ yields another expression $E''$.
\begin{align*}
  x[x \mapsto E] &:= E \\
  y[x \mapsto E] &:= y \quad (\text{if } y \ne x) \\
  (E_1 \,\,\, E_2) [x \mapsto E] &:= (E_1 [x \mapsto E] \,\,\, E_2 [x \mapsto E]) \\
  (\lambda x, \, E) [x \mapsto E'] &:= \lambda x, \, E \\ 
  (\lambda y, \, E) [x \mapsto E'] &:= \lambda y, \, E[x \mapsto E'] \quad (\text{if } y \notin \FV(E')) \\ 
  (\lambda y, \, E) [x \mapsto E'] &:= \lambda z, E[y \mapsto z][x \mapsto E'] \\
                 &\quad \quad \quad (\text{if } z \notin FV(E) \cup FV(E') \cup \{x\}) \\
  (\text{let } y := E \text{ in } E'') [x \mapsto E'] &:= 
    ((\lambda y, \, E) \,\, E'') [x \mapsto E']
\end{align*}

We also define $(\lambda \, (x : T), \, E) [x \mapsto E']$
similar to the case without the optional type annotation above. The only
difference here is we also need to substitute $x$ for $E'$ in the type annotation,
$T$. Note that we do not treat $x$ as being bound in $T$ here.

\begin{align*}
  (\Pi_{x : A} B(x)) [x \mapsto B'] &:= \Pi_{x : A[x \mapsto B']} B(x) \\ 
  (\Pi_{y : A} B(y)) [x \mapsto B'] &:= \Pi_{y : A[x \mapsto B']} B[x \mapsto B'] \quad (\text{if } y \notin \FV(B')) \\ 
  (\Pi_{y : A} B(y)) [x \mapsto B'] &:= \Pi_{z : A[x \mapsto B']} B[y \mapsto z][x \mapsto B'] \\
                 &\quad \quad \quad (\text{if } z \notin FV(B) \cup FV(B') \cup \{x\})
\end{align*}

For Pi expressions, $A$ is an expression and thus when substituting $x$ by
$B'$, we must also perform the substitution in the input type $A$. However, as
$x$ is not bound there, we need not worry about capturing it when substituting there. 

For Sigma expressions, we define $(\Sigma_{x : A} B(x)) [x \mapsto B']$ in a
similar fashion as with the case of Pi.

The substitution rules for \verb|fst|, \verb|snd|, \verb|inl| and \verb|inr| are
trivial and thus ommitted.

Finally, we avoid specifying substitution formally for match expressions since it's
tedious. The key thing to note is that in the expression
$\match(E, x \rightarrow E_1, y \rightarrow E_2)$, $x$ is bound in $E_1$ and $y$
is bound in $E_2$.

\section{Semantics}
\begin{comment}
  https://www.andres-loeh.de/LambdaPi/LambdaPi.pdf
  http://math.andrej.com/2012/11/08/how-to-implement-dependent-type-theory-i/
  
  http://fsl.cs.illinois.edu/images/archive/b/b3/20110221180817!CS522-Spring-2011-PL-book-bigstep.pdf
  https://www.cs.cornell.edu/courses/cs4110/2010fa/lectures/lecture03.pdf
\end{comment}

Now we define a static and big-step semantics for our language.

\subsection{Expressions}
\subsubsection{Overview of reduction and normalization}

In the simply typed lambda calculus, there is only the simple function 
type and lambda abstraction. There, normalization is the process in which 
expressions are reduced through repeated applications of the beta reduction rule
into a form in which no further beta reductions can occur anywhere in the 
expression, even underneath the outermost lambda.
The resulting, reduced expression is said to be in \textit{head normal form}. 
In this section, we want to define similar notions for our richer language.

First, note that we have more than just function types in our language. In
particular, we have 3 main types, namely \verb|Pi|, \verb|Sigma| and \verb|Sum|.
Also, these ``types'' themselves are expressions too. Thus we need to generalize
the notion of normalization to account for these new types and their elimination
rules. Intuitively, we want to repeatedly reduce an expression using all of
these rules until we can reduce no further.

Next note that we will be reducing expressions and substituting with
respect to a context, which contains definitions which the user declares
globally, via \verb|def| and \verb|axiom| statements.

Since \verb|axiom| statements introduce variables at the global level with a
type but no binding, expressions can now contain free variables.
Hence we also need a definition of head normal form and reduction
that takes care of expressions with free variables.
In the next section, we define the head normal form of an expression by way of
a new concept -- the \textit{neutral expression}.
After that, we introduce the \textit{context}, ie environment, before 
discussing normalization.

% Thereafter, we will introduce the notion of a \textit{context}, which we will use
% to define the process of normalization.

% Intuitively, an expression is in head normal form if cannot be further reduced
% by way of an elimination rule. Neutral expressions are those in which the
% elimination rule cannot be applied to reduce an expression because the
% leading term is a free variable.

% Normalization is then the process in which we repeatedly reduce an expression
% using these elimination rules to obtain a head normal form. 
% Note that in our language, we are normalizing with respect to a context, 
% ie we have the possibility to substitute free variables with globally
% defined bindings from the context.

% In the rules to follow, we shall see that this head normal form of an expression,
% should it exist, is unique, just as in the simply typed lambda calculus.

\subsubsection{Neutral expressions, head normal form, beta equivalence}
Informally, neutral expressions are those which cannot be reduced via one of the 
elimination rules (ie the ones for \verb|Pi|, \verb|Sigma| and \verb|Sum| types)
because  the expression to be eliminated is a free variable rather than an
expression of the appropriate type.

With this, we generalize the notion of head normal form to be one in which
the expression cannot be further simplified by applying any of the 3
elimination rules corresponding to the aforementioned types.

Formally we define, using mutual recursion, the subset of neutral and 
normalized expressions via the following judgments:
\begin{enumerate}
  \item $\hnf(E)$ \\
    This asserts that $E$ is an expression that is in head normal form.
  
  \item $\neutral(E)$ \\
    This asserts that $E$ is a neutral expression.
\end{enumerate}

The rules are as follows:
\begin{enumerate}
  \item \[
    \begin{prooftree}
      \hypo{\neutral(E)}
      \infer1{\hnf(E)}
    \end{prooftree}
  \]

  \item \[
    \begin{prooftree}
      \hypo{\hnf(E)}
      \infer1{\hnf(\lambda x, \, E)}
    \end{prooftree}
  \]

  \item \[
    \begin{prooftree}
      \hypo{\hnf(E_1)}
      \hypo{\hnf(E_2)}
      \infer2{\hnf(\Pi_{x : E_1} E_2)}
    \end{prooftree}
  \]

 \item \[
    \begin{prooftree}
      \hypo{\hnf(E_1)}
      \hypo{\hnf(E_2)}
      \infer2{\hnf(\Sigma_{x : E_1} E_2)}
    \end{prooftree}
  \]

 \item \[
    \begin{prooftree}
      \hypo{\hnf(E_1)}
      \hypo{\hnf(E_2)}
      \infer2{\hnf(E_1 + E_2)}
    \end{prooftree}
  \]

  \item \[
    \begin{prooftree}
      \infer0{\neutral(x)}
    \end{prooftree}
  \]

  \item \[
    \begin{prooftree}
      \hypo{\neutral(E_1)} 
      \hypo{\hnf(E_1)} 
      \infer2{\neutral(E_1 \,\, E_2)}
    \end{prooftree}
  \]

  \item \[
    \begin{prooftree}
      \hypo{\neutral(E)} 
      \hypo{\hnf(E_1)} 
      \hypo{\hnf(E_2)} 
      \infer3{\neutral(\match(E, \, x \rightarrow E_1, \, y \rightarrow E_2))}
    \end{prooftree}
  \]

  \item \[
    \begin{prooftree}
      \hypo{\neutral(E)} 
      \infer1{\neutral(\fst \, E)}
    \end{prooftree}
  \]

  \item \[
    \begin{prooftree}
      \hypo{\neutral(E)} 
      \infer1{\neutral(\snd \, E)}
    \end{prooftree}
  \]
\end{enumerate}

Note that \verb|let ... in ...| expressions are not considered to be in head normal form
because we treat expressions of the form \verb|let x := E in E'| as syntactic
sugar for the application $(\lambda x, E') \, E$ during the process of
normalization.

In fact, we will see later that the typing rules for \verb|let|
expressions are also derived using this syntactic trick. 

Finally, we also define the notion of beta equivalence. 

\begin{definition} [Beta equivalence]
We say that two expressions are beta equivalent to each other, written
$E_1 \equiv_\beta E_2$ if they have the same head normal form.

If both expressions are also alpha equivalent, ie
they're equal up to renaming of bound variables, we write
$E_1 \equiv_{\alpha \beta} E_2$.
\end{definition}

We now introduce 2 more types of metavariables corresponding to the new definitions
given in the previous section. 

\begin{enumerate}
\item $\nu, \, \tau$ range over normalized expressions, ie those in head normal
form.
\item $n$ ranges over all neutral terms.
\end{enumerate}

\subsubsection{A brief overview of contexts}
We define contexts, denoted by the metavariable, $\Gamma$, to be lists of
triples of the form
\begin{align*}
  (\text{variable name}, \, \text{type of variable}, \, \text{binding})
\end{align*}

We will use $\emptyset$ to denote the empty context, and $::$ to refer to the
list cons operation.

The binding can be an expression or a special undefined value, which we denote by
\verb|und|.
Contexts have global scope and the top level statements \verb|def|
and \verb|axiom|, return new contexts with updated bindings.
The special \verb|und| value is used for the bindings created by \verb|axiom|
statements and when we want to add a binding for type checking purposes.
In such scenarios, we don't particularly care about the actual binding. We're
only interested in the type of the variable. 

We should mention that whenever we write, say
$(x, \, \tau, \, \nu) \in \Gamma$, we allow the metavariable $\nu$ to be 
\verb|und| as well. Also if there are multiple occurrences of $x$ in $\Gamma$,
as is the case when there is variable shadowing, we refer to the first occurrence
of $x$.

Most of the time, when we work with contexts, we want them to be \textit{well
formed} in the sense that we want the contexts that we deal with in our typing
and normalization judgments to satisfy some additional invariants.

Primarily, we want the ``types'' that we store in the context to be ``valid types"
rather than just being expressions in our language.
We will discuss this in more detail later, alongside the static semantics.

\subsubsection{Overview of judgement forms}
Since we will be defining some judgment forms (including normalization and
typing related ones) in a mutually recursive fashion, we first give an
overview of them.

\begin{enumerate}
\item $\wf(\Gamma)$ \\
  This asserts that a context $\Gamma$ is well formed.

\item $\Gamma \vdash E \Leftarrow T$ and $\Gamma \vdash E \Rightarrow T$ \\
  These judgments will be used to formalize our bidirectional typechecking
  algorithm. They form the typing rules for our language.
  For now, it suffices to say that $\Gamma \vdash E \Leftarrow T$ formalizes the
  meaning that given an expression $E$ and some type $T$, we may verify that
  $E$ has type $T$ under the context $\Gamma$.

  On the other hand, $\Gamma \vdash E \Rightarrow T$ formalizes the notion of
  \textit{type inference}. It says that from a context $\Gamma$, we may infer
  the type of $E$ to be $T$. 
  
  It should be noted here that this isn't real type
  inference using constraint solving and unification. It's a form of lightweight
  type inference that can infer simple stuff like the return type of a function
  application but not the type parameter in the polymorphic identity function
  $\lambda \, (T : \Type) (x : T), \, x$.

\item $\Gamma \vdash E \Downarrow \nu$ \\
  The $\cdot \vdash \cdot \Downarrow \cdot$ relation is our big-step normalization
  process.
  It says that with respect to a context $\Gamma$ containing global bindings, we
  may normalize $E$ to an expression, $\nu$ that is in head normal form.
  % Note that we can do so because our language is a lambda calculus that does
  % not contain full blown recursion and is strongly normalizing. 
\end{enumerate}

\subsubsection{Big step normalization}
% Before defining our type system given by the 2 judgments,
% $\cdot \vdash \cdot \Leftarrow \cdot$ and $\cdot \vdash \cdot \Rightarrow
% \cdot$, we first define $\cdot \vdash \cdot \Downarrow \cdot$, the big
% step semantics for normalizing expressions. This is because we will actually
% need to perform normalization while typechecking.

As mentioned earlier, the aim of normalization is to reduce an
expression, through repeated applications of the 3 elimination rules
(ie \verb|Pi|, \verb|Sigma| and \verb|Sum|) to head normal form, which includes
neutral expressions and free variables as a subset.

One may notice that this intuitive formulation resembles a small-step semantics.
Indeed, we may obtain a contraction rule from each elimination rule and then
formalize a one-step transition relation, say $\Gamma \vdash E \longmapsto E'$.
From that, we can then derive its reflexive, transitive closure,
$\Gamma \vdash E \longmapsto^* E'$.

However we take a different approach. We define a big-step semantics
directly because this translates nicer into a recursive interpreter written in a
functional style.

In the rules below, note how our elimination rules account for expressions
in head normal form, ie $\nu$ and $\tau$, as well as neutral ones, ie $n$.

% It should be noted at this point that in our semantics to follow, we only 
% normalize terms after we typecheck them. 
% In the case of expressions that denote types, this means that we always check
% that they are well formed before performing any computations on them.

% Since \verb|Kind| has no type, we shall not need to define what it means to
% normalize it. 

\begin{enumerate}
\item \textbf{Type}
  \[
    \begin{prooftree}
      \hypo{\wf(\Gamma)}
      \infer1{\Gamma \vdash \Type \Downarrow \Type} 
    \end{prooftree}
  \]
\item \textbf{Variables} \\
  \[
    \begin{prooftree}
      \hypo{\wf(\Gamma)}
      \hypo{(x, \, \tau, \, \nu) \in \Gamma}
      \infer2{\Gamma \vdash x \Downarrow \nu}
    \end{prooftree} 
  \]

  \[
    \begin{prooftree}
      \hypo{\wf(\Gamma)}
      \hypo{(x, \, \tau, \, \text{und}) \in \Gamma}
      \infer2{\Gamma \vdash x \Downarrow x}
    \end{prooftree} 
  \]

\item \textbf{Type ascriptions} \\
  \[
    \begin{prooftree}
      \hypo{\Gamma \vdash E \Downarrow \nu}
      \infer1{\Gamma \vdash (E : T) \Downarrow \nu}
    \end{prooftree}
  \]
  This says type ascriptions do not play a role in computation. They're just
  there to help the type checker figure things out. Users can also use these as
  a form of documentation.
 
\item \textbf{Pi elimination, ie function application} \\
  \[
    \begin{prooftree}
      \hypo{\Gamma \vdash E_1 \Downarrow \lambda x, \, \nu_1}
      \hypo{\Gamma \vdash E_2 \Downarrow \nu_2}
      \hypo{\Gamma \vdash \nu_1[x \mapsto \nu_2] \Downarrow \nu}
      \infer3{\Gamma \vdash E_1 \,\, E_2 \Downarrow \nu}
    \end{prooftree}
  \]

  \[
    \begin{prooftree}
      \hypo{\Gamma \vdash E_1 \Downarrow n}
      \hypo{\Gamma \vdash E_2 \Downarrow \nu}
      \infer2{\Gamma \vdash E_1 \,\, E_2 \Downarrow n \,\, \nu}
    \end{prooftree}
  \]

\item \textbf{Normalizing under lambdas} \\
  \[
    \begin{prooftree}
      \hypo{\Gamma \vdash E \Downarrow \nu}
      \infer1{\Gamma \vdash \lambda x, \, E, \, \Downarrow \lambda x, \, \nu}
    \end{prooftree}
  \]
  \[
    \begin{prooftree}
      \hypo{\Gamma \vdash \lambda x, \, E \Downarrow \nu}
      \infer1{\Gamma \vdash \lambda \, (x : T), \, E, \, \Downarrow \nu}
    \end{prooftree}
  \]
  This second rule says that optional type ascriptions do not play a role in
  normalization, ie we just ignore them.

\item \textbf{Pi type constructor} \\
  \[
    \begin{prooftree}
      \hypo{\Gamma \vdash A \Downarrow \tau}
      \hypo{\Gamma \vdash B(x) \Downarrow \tau'(x)}
      \infer2{\Gamma \vdash \Pi_{x : A} B(x) \Downarrow \Pi_{x : \tau} \tau'(x)}
    \end{prooftree}
  \]

\item \textbf{Local let binding} \\
    \[
      \begin{prooftree}
        \hypo{\Gamma \vdash E \Downarrow \nu}
        \hypo{\Gamma \vdash E'[x \mapsto \nu] \Downarrow \nu'}
        \infer2{\Gamma \vdash \text{let } x := E \text{ in } E' \Downarrow \nu'}
      \end{prooftree}
    \]

\item \textbf{Normalizing under pair constructor} \\
    \[
      \begin{prooftree}
        \hypo{\Gamma \vdash E_1 \Downarrow \nu_1}
        \hypo{\Gamma \vdash E_2 \Downarrow \nu_2}
        \infer2{\Gamma \vdash \langle E_1, \, E_2 \rangle \Downarrow \langle
          \nu_1, \, \nu_2 \rangle}
      \end{prooftree}
    \]

  \item \textbf{Sigma type constructor} \\
    \[
      \begin{prooftree}
        \hypo{\Gamma \vdash A \Downarrow \tau}
        \hypo{\Gamma \vdash B \Downarrow \tau'(x)}
        \infer2{\Gamma \vdash \Sigma_{x : A}B(x) \Downarrow \Sigma_{x : \tau}\tau'(x)}
      \end{prooftree}
    \]

  \item \textbf{Sigma elimination} \\
   \[
    \begin{prooftree}
      \hypo{\Gamma \vdash E \Downarrow (\nu_1, \, \nu_2)}
      \infer1{\Gamma \vdash \fst E \Downarrow \nu_1}
    \end{prooftree}
  \]

  \[
   \begin{prooftree}
    \hypo{\Gamma \vdash E \Downarrow n}
     \infer1{\Gamma \vdash \fst E \Downarrow \fst n}
   \end{prooftree}
  \]

  \[
    \begin{prooftree}
      \hypo{\Gamma \vdash E \Downarrow (\nu_1, \, \nu_2)}
      \infer1{\Gamma \vdash \snd E \Downarrow \nu_2}
    \end{prooftree}
  \]

  \[
   \begin{prooftree}
     \hypo{\Gamma \vdash E \Downarrow n}
     \infer1{\Gamma \vdash \snd E \Downarrow \snd n}
   \end{prooftree}
  \]

  % \[
  %   \begin{prooftree}
  %     \hypo{\Gamma \vdash E \Downarrow \langle \nu_1, \, \nu_2 \rangle}
  %     \hypo{\Gamma \vdash E'[x_1 \mapsto \nu_1][x_2 \mapsto \nu_2] \Downarrow \nu}
  %     \infer2{\Gamma \vdash \text{let } (x_1, \, x_2) := E \text{ in } E' \Downarrow \nu}
  %   \end{prooftree}
  % \]

  \item \textbf{Sum type constructor}
  \[
    \begin{prooftree}
      \hypo{E_1 \Downarrow \nu_1}
      \hypo{E_2 \Downarrow \nu_2}
      \infer2{E_1 + E_2 \Downarrow \nu_1 + \nu_2}
    \end{prooftree}
  \]

  \item \textbf{Normalizing under sum data constructors}
  \[
    \begin{prooftree}
      \hypo{\Gamma \vdash E \Downarrow \nu}
      \infer1{\Gamma \vdash \inl \, E \Downarrow \inl \, \nu}
    \end{prooftree}  
  \]

  \[
    \begin{prooftree}
     \hypo{\Gamma \vdash E \Downarrow \inl \nu}
     \infer1{\Gamma \vdash \inr \, E \Downarrow \nu}
    \end{prooftree}  
  \]

  \item \textbf{Normalizing sum eliminator}
  \[
    \begin{prooftree}
      \hypo{\Gamma \vdash E \Downarrow \inl \nu}
      \hypo{\Gamma \vdash E_1[x \mapsto \nu] \Downarrow \nu_1}
      \infer2{\Gamma \vdash 
        \match(E, \, x \rightarrow E_1, \, y \rightarrow E_2)
        \Downarrow \nu_1}
    \end{prooftree}  
  \]

  \[
   \begin{prooftree}
     \hypo{\Gamma \vdash E \Downarrow \inr \nu}
     \hypo{\Gamma \vdash E_2[y \mapsto \nu] \Downarrow \nu_2}
     \infer2{\Gamma \vdash 
       \match(E, \, x \rightarrow E_1, \, y \rightarrow E_2)
       \Downarrow E_2[y \mapsto \nu_2]}
   \end{prooftree}  
  \]

  \[
   \begin{prooftree}
    \hypo{\Gamma \vdash E \Downarrow n}
      \hypo{\Gamma \vdash E_1 \Downarrow \nu_1}
      \hypo{\Gamma \vdash E_2 \Downarrow \nu_2}
      \infer3{\Gamma \vdash \match(E, \, x \rightarrow E_1, \, y \rightarrow E_2) 
               \Downarrow
               \match(n, \, x \rightarrow \nu_1, \, y \rightarrow \nu_2) 
               }
   \end{prooftree}  
  \]

\end{enumerate}

Note that the lack of a normalization rule for \verb|Kind| is deliberate.
The reason is that we will ensure that we only normalize expressions after we
typecheck them, and we will see in the next section that \verb|Kind| has no
type. Thus we will never need to normalize it.

\subsubsection{Well formed types and contexts}
Recall that when defining the notion of a context earlier, we said that we
normally only work with well formed ones. In particular, this means that the
types we store in the context are ``valid''. Here we properly define these
concepts.

First we introduce a new metavariable $s$ to denote either of the 2 sorts,
\verb|Type| and \verb|Kind|. 

\begin{definition} [Well formed type and type constructor]
  An expression $T$ is said to be a well-formed type with respect to the context
  $\Gamma$ if it satisfies
  \[ \Gamma \vdash T \Leftarrow s \]

  Informally, we can think of $\Type$ as the ``type of all small types'' and so
  all small well formed types are those expressions satisfying
  $\Gamma \vdash T \Leftarrow \Type$, ie they can be checked to have type $\Type$.

  Those $T$ satisfying $\Gamma \vdash T \Leftarrow \Kind$ instead represent the
  type of type constructors. In other words, the type of a type constructor is a
  Kind, just like in Haskell. 
\end{definition}

With this, we define well formed contexts as contexts satisfying 2 key invariants:
\begin{enumerate}
  \item Well formed contexts only contain well formed types.
  \item The expressions representing the type and binding of a variable are both
  in head normal form.
\end{enumerate}

Observe that in order to maintain these 2 invariants while typechecking
an expression,  we will need to perform normalization on types themselves.
To see this, consider what happens when we want to typecheck a lambda abstraction.
Intuitively, we procceed as in the simply typed lambda calculus. We want to add
the variable and its type to the context and continue typechecking the body.
But remember, we must ensure that the context remains well formed when
typechecking the body. For this, we must check that the type is well formed and
normalize it to head normal form before we can add it to the context.

However, it must be noted that we only normalize expressions after we typecheck
them. At no point in our formulation in the next section, do we attempt to
normalize an expression before checking that it has a well formed type.

\begin{definition} [Well formed context]
  Recalling that $\emptyset$ denotes the empty list and that $::$ denotes the list
  cons operation, we define precisely the judgment $\wf(\Gamma)$.

  \begin{enumerate}
  \item \textbf{Base case} \\
    \[
      \begin{prooftree}
        \hypo{\wf(\emptyset)}
      \end{prooftree}
    \]

  \item \textbf{Inductive cases} \\
    \[
      \begin{prooftree}
        \hypo{\wf(\Gamma)}
        \hypo{\Gamma \vdash \tau \Leftarrow s}
        \infer2{\wf ((x, \, \tau, \, \nu) :: \Gamma)}
      \end{prooftree}
    \]

    \[
      \begin{prooftree}
        \hypo{\wf(\Gamma)}
        \hypo{\Gamma \vdash \tau \Leftarrow s}
        \infer2{\wf ((x, \, \tau, \, \text{und}) :: \Gamma)}
      \end{prooftree}
    \]
  \end{enumerate}
\end{definition}

\subsubsection{Static semantics, ie typing rules}
\begin{definition} [Bidrectional typechecking]
  Here we define the 2 \textit{mutually recursive} relations
  \begin{enumerate}
  \item$\cdot \vdash \cdot \Rightarrow \cdot $ which corresponds to \textit{inference}
  \item$\cdot \vdash \cdot \Leftarrow \cdot$ which corresponds to \textit{checking}
  \end{enumerate}

  The idea is that there are some expressions for which it is easier to
  \textit{infer},
  ie compute the type directly, while for others, it is easier to have the user
  supply a type annotation and then \textit{check} that it is correct.

  As a rule of thumb, it is often easier to check the type for
  introduction rules while for elimination rules, it is usually easier to infer
  the type.

  \begin{enumerate}
  \item \textbf{Var} \\
    \[
      \begin{prooftree}
        \hypo{\wf(\Gamma)}
        \hypo{(x, \, \tau, \, v) \in \Gamma}
        \infer2{\Gamma \vdash x \Rightarrow \tau}
      \end{prooftree}
    \]
    This says that we may infer the type of a variable if the type information
    is already in our context.

  \item \textbf{Type} \\
    \[
      \begin{prooftree}
        \hypo{\wf(\Gamma)}
        \infer1{\Gamma \vdash \Type \Rightarrow \Kind}
      \end{prooftree}
    \]

    % \[
    %   \begin{prooftree}
    %     \hypo{\wf(\Gamma)}
    %     \infer1{\Gamma \vdash \Type \Rightarrow \Type}
    %   \end{prooftree}
    % \]

  \item \textbf{Type ascriptions} \\
    \[
      \begin{prooftree}
        \hypo{\Gamma \vdash T \Rightarrow s}
        \hypo{\Gamma \vdash T \Downarrow \tau}
        \hypo{\Gamma \vdash E \Leftarrow \tau}
        \infer3{\Gamma \vdash (E : T) \Rightarrow \tau}
      \end{prooftree}
    \]
    Optional type ascriptions allow the interpreter to infer the type of an
    expression. This is useful for lambda abstractions in particular because it's
    kinda hard to infer the type of a function like $\lambda x, \, x$ without any
    further contextual information.

    \[
      \begin{prooftree}
        \hypo{\Gamma \vdash E \Leftarrow \Kind}
        \infer1{\Gamma \vdash (E : \Kind) \Rightarrow \Kind}
      \end{prooftree}
    \]
    Note that the first rule doesn't allow users to assert that $(E : \Kind)$
    since there is no $s$ with $\Gamma \vdash \Kind \Rightarrow s$.
    This second rule allows users to assert that \verb|Type| and type constructors
    have type \verb|Kind|.
    
  \item \textbf{Check} \\
    \[
      \begin{prooftree}
        \hypo{\Gamma \vdash E \Rightarrow \tau'}
        \hypo{\tau \equiv_{\alpha \beta} \tau'}
        \infer2{\Gamma \vdash E \Leftarrow \tau}
      \end{prooftree}
    \]

    This says that to check if $E$ has type $\tau$ with respect to a context
    $\Gamma$, we may first infer the type of $E$. Suppose it is $\tau'$. Then
    if we also find that $\tau$ and $\tau'$ are $\alpha$ and $\beta$ equivalent
    to each other, we may conclude that $E$ indeed has type $\tau$.

    \begin{remark}
      This rule, together with the one on type ascriptions, is precisely the reason
      why we eagerly normalize types and maintain the 2 invariants required
      for our contexts to be well formed.

      To see this, suppose the user requests that we typecheck an expression
      of the form $(x : T)$
      where $T$ is some complicated expression entered by the user.
      Intuitively, we want to grab the type of $x$ from the context and then
      check that it is equal to the annotated type of $T$.

      Unlike the simply typed and polymorphic lambda calculi, our type system is
      much richer and so checking types for equality is not so simple.
      In fact, ``types'' are just expressions!

      Ah but since we always normalize types before dealing with them, assuming
      that terms have a unique head normal form 
      (we'll discuss this later in the section on metatheoretic properties), we 
      will normalize $T$ and $T'$ to $\tau$ and $\tau'$ respectively.
      After that, we need only compare them for structural equality,
      modulo alpha equivalence. Such an operation is trivial to implement using
      an implementation that represets variables by de bruijn indices.
    \end{remark}

  \item \textbf{Pi formation} \\
    \[
      \begin{prooftree}
        \hypo{\Gamma \vdash A \Rightarrow s_1}
        \hypo{\Gamma \vdash A \Downarrow \tau}
        \hypo{(x, \, \tau, \, \text{und}) :: \Gamma \vdash B(x) \Rightarrow s_2}
        \infer3{\Gamma \vdash \Pi_{x : A}B(x) \Rightarrow s_2}
      \end{prooftree}
    \]
    Note that this is a rule schema with the metavariables 
    $s_1, \, s_2 \in \{\Type, \, \Kind\}$.

    % \[
    %   \begin{prooftree}
    %     \hypo{\Gamma \vdash A \Leftarrow \Type}
    %     \hypo{\Gamma \vdash A \Downarrow \tau}
    %     \hypo{(x, \, \tau, \, \text{undefined}) :: \Gamma \vdash B(x) \Leftarrow \Type}
    %     \infer3{\Gamma \vdash \Pi_{x : A}B(x) \Rightarrow \Type}
    %   \end{prooftree}
    % \]

  \item \textbf{Pi introduction} \\
    \[
      \begin{prooftree}
        \hypo{(x, \, \tau, \, \text{und}) :: \Gamma \vdash E \Leftarrow \tau'(x)}
        \infer1{\Gamma \vdash \lambda x, \, E \Leftarrow \Pi_{x : \tau}\tau'(x)}
      \end{prooftree}
    \]
    Note that in the event that the input variable of the lambda abstraction and
    Pi are different, then we must perform an $\alpha$ renaming so that the
    variable being bound in $(\lambda x, \, E)$ and $\Pi_{y : \tau} \tau'(y)$
    are the same.

    \[
      \begin{prooftree}
        \hypo{\Gamma \vdash T \Rightarrow s}
        \hypo{\Gamma \vdash T \Downarrow \tau}
        \hypo{(x, \, \tau, \, \text{und}) :: \Gamma \vdash E \Rightarrow \tau'(x)}
        \infer3{\Gamma \vdash \lambda \, (x : T), \, E \Rightarrow \Pi_{x : \tau}\tau'(x)}
      \end{prooftree}
    \]
    The second rule says that if the user type annotates the input argument of
    the function, then we can try to infer the type of the output and
    consequently, the type of the function as a whole.

  \item \textbf{Function application, ie Pi elimination} \\
    \[
      \begin{prooftree}
        \hypo{\Gamma \vdash E_1 \Rightarrow \Pi_{x : \tau}\tau'(x)}
        \hypo{\Gamma \vdash E_2 \Leftarrow \tau}
        \hypo{\Gamma \vdash \tau'[x \mapsto E_2] \Downarrow \tau''}
        \infer3{E_1 \,\, E_2 \Rightarrow \tau''}
      \end{prooftree}
    \]

  \item \textbf{Local let binding} \\
    \[
      \begin{prooftree}
        \hypo{\Gamma \vdash E \Rightarrow \tau}
        \hypo{(x, \, \tau, \, \text{und}) :: \Gamma \vdash E' \Rightarrow \tau'(x)}
        \hypo{\Gamma \vdash \tau'[x \mapsto E] \Downarrow \tau''}
        \infer3{\Gamma \vdash \text{let } x := E \text{ in } E' \Rightarrow \tau''}
      \end{prooftree}
    \]

  \item \textbf{Sigma formation} \\
    \[
      \begin{prooftree}
        \hypo{\Gamma \vdash A \Rightarrow s_1}
        \hypo{\Gamma \vdash A \Downarrow \tau}
        \hypo{(x, \, \tau, \, \text{und}) :: \Gamma \vdash B(x) \Rightarrow s_2}
        \infer3{\Gamma \vdash \Sigma_{x : A}B(x) \Rightarrow s_2}
      \end{prooftree}
    \]

    As with the rule for Pi formation, $s_1, \, s_2 \in \{\Type, \, \Kind\}$.

  \item \textbf{Sigma introduction} \\
    \[
      \begin{prooftree}
        \hypo{\Gamma \vdash E_1 \Leftarrow \tau_1}
        \hypo{\Gamma \vdash \tau_2[x \mapsto E_1] \Downarrow \tau_2'}
        \hypo{\Gamma \vdash E_2 \Leftarrow \tau_2'}
        \infer3{\Gamma \vdash \langle  E_1, \, E_2 \rangle \Leftarrow 
          \Sigma_{x : \tau_1}\tau_2(x)}
      \end{prooftree}
    \]
   
  \item \textbf{Sigma elimination} \\
    \[
      \begin{prooftree}
        \hypo{\Gamma \vdash E \Rightarrow \Sigma_{x : \tau_1}\tau_2(x)}
        \infer1{\Gamma \vdash \fst E \Rightarrow \tau_1}
      \end{prooftree}
    \]

    \[
      \begin{prooftree}
        \hypo{\Gamma \vdash E \Rightarrow \Sigma_{x : \tau_1}\tau_2(x)}
        \hypo{\Gamma \vdash \tau_2[x \mapsto \fst E] \Downarrow \tau_2'}
        \infer2{\Gamma \vdash \snd E \Rightarrow \tau_2'}
      \end{prooftree}
    \]

    % \[
    %   \begin{prooftree}
    %     \hypo{E \Rightarrow \Sigma_{x : \tau_1}\tau_2(x)}
    %     \hypo{\Gamma \vdash \tau_2[x \mapsto \fst E] \Downarrow \tau_2'}
    %     \hypo{(y, \, \tau_2', \, \text{und}) :: (x, \, \tau_1, \, \text{und}) :: \Gamma
    %           \vdash E' \rightarrow \tau}
    %     \infer3{\Gamma \vdash \text{let } (x, \, y) := E \text{ in } E' \Rightarrow \tau}
    %   \end{prooftree} 
    % \]

    % \item \textbf{Pair destructuring} \\
    % \[
    %   \begin{prooftree}
    %     \hypo{\Gamma \vdash E \Rightarrow \Sigma_{x : \tau_1}\tau_2(x)}
    %     % \hypo{(x_2, \, \tau_2[x \mapsto x_1], \text{und}) :: 
    %     %   (x_1, \, \tau_1, \, \text{und}) :: 
    %     %   \Gamma \vdash E' \Rightarrow \tau(x_1, \, x_2)}
    %     \hypo{\Gamma \vdash E' [x_1 \mapsto \fst E][x_2 \mapsto \snd E]
    %       \Rightarrow \tau}
    %     \infer2{\Gamma \vdash (\text{let } (x_1, \, x_2) := E
    %       \text{ in } E') \Rightarrow \tau}
    %   \end{prooftree}
    % \]

    \item \textbf{Sum formation} \\
    \[
      \begin{prooftree}
       \hypo{\Gamma \vdash A \Leftarrow \Type} 
       \hypo{\Gamma \vdash B \Leftarrow \Type} 
       \infer2{\Gamma \vdash A + B \Rightarrow \Type} 
      \end{prooftree}
    \]
    Note that this rule says that users can only make construct a sum, aka 
    coproduct, out of types that live in the universe \verb|Type|, not 
    \verb|Kind|.

    Recalling the Curry-Howard correspondence which identifies types and
    propositions, we see sum types as a way to model disjunction. Hence we do
    not see a need to allow users to construct sums out of large types like
    type constructors found in \verb|Kind|.

    Also we are unsure if the logical consistency of the system can be preserved
    if we allow for $A + B$ to be formed when one is of type \verb|Type| while
    the other has type \verb|Kind|. 

    \item \textbf{Sum introduction} \\
    \[
      \begin{prooftree}
       \hypo{\Gamma \vdash E \Leftarrow \tau_1} 
       \infer1{\Gamma \vdash \inl E \Leftarrow \tau_1 + \tau_2} 
      \end{prooftree}
    \]

    \[
     \begin{prooftree}
      \hypo{\Gamma \vdash E \Leftarrow \tau_2} 
      \infer1{\Gamma \vdash \inr E \Leftarrow \tau_1 + \tau_2} 
     \end{prooftree}
   \]

   \item \textbf{Sum elimination} \\
    \[
    \begin{prooftree}
      \hypo{\Gamma \vdash E \Rightarrow \tau_1 + \tau_2}
      \hypo{(x, \, \tau_1, \, \text{und}) :: \Gamma \vdash E_1 \Rightarrow \tau_1'}
      \hypo{(y, \, \tau_2, \, \text{und}) :: \Gamma \vdash E_2 \Rightarrow \tau_2'}
      \hypo{\tau_1' \equiv_{\alpha \beta} \tau_2'}
      \infer4{\Gamma \vdash \match(E, \, x \rightarrow E_1, \, y \rightarrow E_2)
               \Rightarrow \tau_1'}
    \end{prooftree}  
  \]

  % \item \textbf{Local let binding} \\
  %   \[
  %     \begin{prooftree}
  %       \hypo{\Gamma \vdash E \Rightarrow \tau}
  %       \hypo{(x, \, \tau, \, \text{undefined}) :: \Gamma \vdash E' \Rightarrow \tau'(x)}
  %       \hypo{\Gamma \vdash \tau'[x \mapsto E] \Downarrow \tau''}
  %       \infer3{\Gamma \vdash (\text{let } x := E \text{ in } E') \Rightarrow \tau''}
  %     \end{prooftree}
  %   \]

  %   \[
  %     \begin{prooftree}
  %       \hypo{\Gamma \vdash (\text{let } x := (E : T)) \text{ in } E' \Rightarrow \tau}
  %       \infer1{\Gamma \vdash (\text{let } (x : T) := E \text{ in } E') \Rightarrow \tau}
  %     \end{prooftree}
  %   \]
  %   This second rule handles the case when an optional type annotation is given
  %   for the variable $x$.

  % \item \textbf{Pair destructuring} \\
  %   \[
  %     \begin{prooftree}
  %       % \hypo{\Gamma \vdash E \Rightarrow \Sigma_{x : \tau_1}\tau_2(x)}
  %       \hypo{\Gamma \vdash (\lambda z, \, E'[x_1 \mapsto \fst z][x_2 \mapsto
  %         \snd z]) \,\, E
  %         \Rightarrow \tau}
  %         \hypo{z \notin \FV(E')}
  %         \infer2{\Gamma \vdash (\text{let } \langle x_1, \, x_2 \rangle := E
  %         \text{ in } E') \Rightarrow \tau}
  %       \end{prooftree}
  %     \]

  \end{enumerate}
\end{definition}

The nice thing about these rules is that we can translate it almost directly
into a typechecking and inference algorithm!
More precisely, we may translate $\Gamma \vdash E \Leftarrow T$ into a
function called $\text{check}(\Gamma, \, E, \, T)$ which checks if the
expression $E$ really has the type $T$ given a context $\Gamma$.

Similarly, $\Gamma \vdash E \Rightarrow T$ gives us the function
$\text{infer}(\Gamma, \, E)$ which outputs the inferred type of $E$ given
the context $\Gamma$.

In the literature, such rules are called \textit{syntax directed}, as the
algorithm closely follows the formalization of the corresponding judgments.

% Another kind of equivalence is \textit{eta equivalence}. To explain this,
% consider the terms $\lambda x, \, B \, x$ and $B$. Both of them have exactly the
% same behavior. This is one form of eta equivalence.
% In lambda calculus terms, the latter is obtained from the former by an eta
% reduction.

% Now, we may view the lambda as the data constructor for
% the Pi type and function application as the eliminator.
% Then the expression $\lambda x, \, B \, x$ is obtained from $B$ by first
% eliminating $B$ to obtain $B x$ and then using the lambda
% constructor to obtain $\lambda x, \, B \, x$.

% More generally, we say that 2 terms are eta equivalent to each other if one can
% be obtained from the other by eliminating the term and then reconstructing the
% original term using the constructor. For instance, if we had a binary coproduct,
% we could project out both components and then recombine them to obtain the
% original sum.

% Unfortunately our type system is intensional rather than extensional in
% that the following rule does \textit{not} hold: 
% \[
%   \begin{prooftree}
%     \hypo{\Gamma \vdash E \Leftarrow \tau}
%     \hypo{\tau =_\eta \tau'}
%     \infer2{\Gamma \vdash E \Leftarrow \tau'}
%   \end{prooftree}
% \]

\subsection{Statements and programs}
\begin{comment}
  Follows this approach:
  https://www.cs.cornell.edu/courses/cs6110/2013sp/lectures/lec05-sp13.pdf
\end{comment}

Programs in our language are nonempty sequences of statements, with each
statement being built from some expression.
We formalize the execution of programs by defining a
\textit{transition system}.

Before that, we first define the notion of a configuration, which plays the
same role as states in automata. 

\subsubsection{Configuration/state of program}
A configuration has the form
\begin{align*}
  (\Gamma, \, \langle S_0; \, \dots; \, S_n \rangle, \, E)
\end{align*}
Configurations are triples representing the instantaneous state during an
execution of a program. Here, $\Gamma$ denotes the current state of the
global context and the sequence $\langle S_0; \, \dots; \, S_n \rangle$ denotes
the sequence of statements to be executed next. 
The expression $E$ is used to indicate the output of the previously executed
statement.

With this view, given a program, $\langle S_0; \, \dots; \, S_n
\rangle$, we also define the initial and final configurations.
\begin{enumerate}
\item \textbf{Initial configuration:}
  $ (\emptyset, \, \langle S_0; \, \dots; \, S_n \rangle, \, \Type)$

  This is the state in which we begin executing our programs.
  Initially, our global context is empty. Also, \verb|Type| is merely a dummy
  value and could have very well be replaced by \verb|Kind|.

\item \textbf{Final configurations} $(\Gamma, \, \langle \rangle, \, E)$
After executing programs in our language, if all goes well, we'll end up in
one of these states.
\end{enumerate}

In the next section, we define the big-step transition relation for programs $\cdot
\Downarrow \cdot$ using
this notion of a configuration.

\subsubsection{Big step semantics for programs}
We define the big step transition relation, $\cdot \Downarrow \cdot$, via a
new judgment form.

It should be noted that in the rules below, we interleave type checking and 
evaluation, ie we type check and then evaluate each statement, one at a time,
carrying the context along as we go.
We do not statically type check the whole program first, then evaluate
afterwards.

\begin{enumerate}
\item \textbf{Check} \\
  \[
    \begin{prooftree}
      \hypo{\Gamma \vdash E \Rightarrow s}
      \infer1{(\Gamma, \, \langle \checkk \, E \rangle, \, E') \Downarrow
        (\Gamma, \, \langle \rangle, \, \tau)}
    \end{prooftree}
  \]

\item \textbf{Eval} \\
  \[
    \begin{prooftree}
      \hypo{\Gamma \vdash E \Rightarrow s}
      \hypo{\Gamma \vdash E \Downarrow \nu}
      \infer2{(\Gamma, \, \langle \evall \, E \rangle, \, E') \Downarrow
        (\Gamma, \, \langle \rangle, \, \nu)}
    \end{prooftree}
  \]

\item \textbf{Axiom} \\
  \[
    \begin{prooftree}
      \hypo{\Gamma \vdash T \Rightarrow s}
      \hypo{\Gamma \vdash T \Downarrow \tau}
      \infer2{(\Gamma, \, \langle \axiom x : T \rangle, \, E) \Downarrow
        ((x, \, \tau, \, \text{und}) :: \Gamma, \, \langle \rangle, \, x)}
    \end{prooftree}
  \]

\item \textbf{Def} \\
  \[
    \begin{prooftree}
      \hypo{\Gamma \vdash E \Rightarrow s}
      \hypo{\Gamma \vdash E \Downarrow \nu}
      \infer2{(\Gamma, \, \langle \deff x := E \rangle, \, E') \Downarrow
        ((x, \, \tau, \, \nu) :: \Gamma, \, \langle \rangle, \, x)}
    \end{prooftree}
  \]

\item \textbf{Sequences of statements} \\
  \[
    \begin{prooftree}
      \hypo{(\Gamma, \, \langle S_0 \rangle, \, E) \Downarrow
        (\Gamma', \, \langle \rangle, \, E') }
      \hypo{(\Gamma', \, \langle S_1; \, \dots; \, S_n \rangle, \, E) \Downarrow
        (\Gamma'', \, \langle \rangle, \, E'') }
      \infer2{(\Gamma, \, \langle S_0; \, S_1; \, \dots; \, S_n \rangle, \, E) \Downarrow
        (\Gamma'', \, \langle \rangle, \, E'') }
    \end{prooftree}
  \]

\end{enumerate}

\subsubsection{Putting the transition system together}
Letting $S$ denote the (infinite) set of all configurations, and defining
\[ F := \{ (\Gamma, \, \langle \rangle, \, E) \in S \, | \, E \text{ expression} \} \]

we obtain a transition system, given by the tuple
\begin{align*}
  \langle S, \, \Downarrow, \, 
  (\emptyset, \, \langle S_0; \, \dots; \, S_n \rangle, \, \Type), \, F \rangle
\end{align*}

Notice how our formalization mimics the definition of an automaton.

\subsubsection{Output expression of evaluating a program}
With these rules, given a user-entered program, say $\langle S_0;
\dots; S_n \rangle$, we define the
output expression of a program to be the $E$ such that
\begin{align*}
  (\emptyset, \, \langle S_0; \, \dots \; S_n \rangle, \, \Type) \Downarrow
  (\Gamma, \, \langle \rangle, \, E)
\end{align*}

In other words, the expression output to the user is the expression obtained by
beginning with the initial configuration and then recursively evaluating
until we reach a final configuration.

\section{Metatheoretic discussion -- TODO}
The original Calculus of Constructions is known to be strongly normalizing in
that the normalization process, when applied to any well typed term, always
terminates. Furthermore, it has decidable typechecking and is logically
consistent when its type system is viewed as a logical calculus, with the Pi type
corresponding to universal quantification.


Say something about why we think our bidirectional typechecking works.
https://arxiv.org/pdf/2102.06513.pdf

Also say something about how ``impredicative'' Sigma types breaks consistency
due to Girard's paradox and that the only (?) way to fix that is to use a 
predicative hierarchy of type universe with either cumulativitiy or universe
polymorphism.

Mention that this issue may not be a big one because the proof term in Girard's
is astronomical and Idk if our system can even handle anything that big since it
isn't very efficient (cos we substitute naively rather than use normalization
by evaluation).

If users are really afraid, just avoid using higher order existential
quantification and stick to first order. Alternatively, can encode higher order
existential quantification in terms of universal in CoC (see Type Theory and
Formal Proof book).

https://era.ed.ac.uk/bitstream/handle/1842/12487/Luo1990.Pdf



Here we make an assumption that head normal forms are \textit{unique}. 
This is not a far fetched assumption because it holds for the original Calculus
of Constructions as well as various extensions of it that include \verb|Sigma|
types among others. [insert reference to Luo's phd thesis]



It's worth noting that $\eta$ equivalence is not respected by our type system.
By that we mean that the following rule does not hold:
\[
  \begin{prooftree}
    \hypo{\Gamma \vdash E \Rightarrow \tau'}
    \hypo{\tau \equiv_{\eta} \tau'}
    \infer2{\Gamma \vdash E \Leftarrow \tau}
 \end{prooftree}
\]
In other words, 2 types that are $\eta$ equivalent to one another will 
\textit{not} be judged as equal types.

One way to fix this is to allow the type checker to eta expand terms, via
eliminating and then applying the data constructor while computing the beta
normal form. However, this significantly complicates discussions of the
metatheoretic properties of the language and so for simplicity, we chose to
follow the original formulation of the Calculus of Constructions and ignore
this.

% http://citeseerx.ist.psu.edu/viewdoc/download?doi=10.1.1.39.9149&rep=rep1&type=pdf
% http://www2.tcs.ifi.lmu.de/~abel/lfsigma.pdf

\section{Technical implementation}

\end{document}