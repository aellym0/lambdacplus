\documentclass{article}
\usepackage[utf8]{inputenc}

\usepackage{amsmath, amssymb, amsthm, thmtools, thm-restate, tikz, graphicx,
  hyperref, cleveref, comment, expl3, xparse, ebproof, enumitem, xcolor,
  stmaryrd, backnaur, verbatim}

\usepackage[titlenumbered, ruled]{algorithm2e}
\usetikzlibrary{shapes.geometric}
\usetikzlibrary{positioning}
\usetikzlibrary{cd}

\hypersetup{
  linktoc=all,
  allbordercolors={0 0 0},
  pdfborderstyle={/S/U/W 1}
}

\usepackage[mathscr]{euscript}

% https://tex.stackexchange.com/questions/37797/theorem-environment-line-break-after-label
\newtheoremstyle{break}% name
{}%         Space above, empty = `usual value'
{}%         Space below
{\itshape}% Body font
{}%         Indent amount (empty = no indent, \parindent = para indent)
{\bfseries}% Thm head font
{.}%        Punctuation after thm head
{\newline}% Space after thm head: \newline = linebreak
{}%         Thm head spec

% https://tex.stackexchange.com/questions/250035/transform-output-theoremstyleremark-from-italics-to-bold
\newtheoremstyle{boldremark}
{\dimexpr\topsep/2\relax} % space above
{\dimexpr\topsep/2\relax} % space below
{}          % body font
{}          % indent amount
{\itshape\bfseries}% theorem head font
{.}         % punctuation after theorem head
{\newline}      % space after theorem head
{}          % theorem hed spec. (empty = "normal")

\newtheoremstyle{discussion}% name
{}%         Space above, empty = `usual value'
{}%         Space below
{}% Body font
{}%         Indent amount (empty = no indent, \parindent = para indent)
{\bfseries}% Thm head font
{.}%        Punctuation after thm head
{\newline}% Space after thm head: \newline = linebreak
{}%         Thm head spec

\theoremstyle{definition}
\newtheorem{definition}{Definition}[section]
\newtheorem{notation}{Notation}[section]

\theoremstyle{break}
\newtheorem{theorem}{Theorem}[section]
\crefname{theorem}{Theorem}{Theorems}

\theoremstyle{break}
\newtheorem{corollary}{Corollary}[theorem]
\crefname{corollary}{Corollary}{Corollaries}

\theoremstyle{break}
\newtheorem{lemma}[theorem]{Lemma}
\crefname{lemma}{Lemma}{Lemmas}

\theoremstyle{remark}
\newtheorem*{remark}{Remark}

% https://latex.org/forum/viewtopic.php?t=24225
\theoremstyle{discussion}
\newtheorem{discussion}[theorem]{Discussion}
\crefname{discussion}{Discussion}{Discussions}

\theoremstyle{boldremark}
\newtheorem*{boldremark}{Remark}

% \newcommand{\powerset}{\mathscr{P}}
\newcommand{\powerset}{\mathbb{P}}
\newcommand{\N}{\mathbb{N}}
\newcommand{\Z}{\mathbb{Z}}
\newcommand{\R}{\mathbb{R}}

\newcommand{\fact}{\mathbf{fact}}

% Partial function arrow
% https://math.stackexchange.com/questions/742614/notation-for-partial-function-set
\newcommand{\pto}{\multimap\mspace{-2mu} \to}

\DeclareMathOperator{\Type}{Type}
\DeclareMathOperator{\FV}{FV}
\DeclareMathOperator{\Pii}{Pi}
\DeclareMathOperator{\foralll}{forall}
\DeclareMathOperator{\fun}{fun}
\DeclareMathOperator{\deff}{def}
\DeclareMathOperator{\axiom}{axiom}
\DeclareMathOperator{\checkk}{check}
\DeclareMathOperator{\evall}{eval}
\DeclareMathOperator{\wf}{wf}
\DeclareMathOperator{\fst}{fst}
\DeclareMathOperator{\snd}{snd}

% \DeclareMathOperator{\lett}{let}
% \DeclareMathOperator{\inn}{in}

% https://math.stackexchange.com/questions/298648/is-there-a-common-symbol-for-concatenating-two-finite-sequences
\newcommand\mdoubleplus{\mathbin{+\mkern-10mu+}}


\begin{document}

\section{Preliminaries}
\subsection{Overview}

BIG TODO: Write something about Pure Type Systems and the lambda cube and how
this language is essentially an extension of the Calculus of Constructions, with
stuff like top level definitions and lightweight Typed Source esque type
inference.

Our language is a dependently typed lambda calculus, based on the calculus of
constructions.
The main references which we base our language on are
\href{https://www.andres-loeh.de/LambdaPi/LambdaPi.pdf}{this paper on $\lambda
  \Pi$} and \href{https://github.com/andrejbauer/spartan-type-theory}{this
  implementation of a dependent type theory}.

In System F, types may depend on other type variables, thus enabling
parametrically polymorphic functions. 
In our language, we generalize this to allow types to depend on terms as well.
In order to achieve this, we promote all types to expressions, so we no longer
have separate syntactic categories for them. 

Yes, this means that \textit{everything} in our language is an expression, even those
things which we call types!
With this in mind, a type is then an expression, say $T$, which satisfies a very
specific typing judgment, ie $\Gamma \vdash T \Leftarrow \Type$. We'll revisit
this again later.

The main aim of our project is to implement a simple lambda calculus with the
dependent function type. These are also known as Pi types in the literature.
These generalize the simple function type $A \to B$ by allowing the output type
$B$ to now \textit{depend on the value} of the input expression.
The type of functions in our language is now written as $\Pi_{x : A} B(x)$ where
we write $B(x)$ for the return type to emphasize that $x$ may appear free in
$B$.

The set theoretic analogue to this dependent Pi type is the generalized
cartesian product. Given a set $A$ and a family of sets $\langle B_x \, | \, x
\in A\rangle$ indexed by the
elements $x \in A$, we can form the generalized cartesian product, denoted
\begin{align*}
  \Pi_{x \in A} B_x = \bigg\{ f : A \rightarrow \bigcup_{x \in A} B_x \, | \, \forall x \in A, \, f(x) \in B_x \bigg\}
\end{align*}
Functions that inhabit this set are known as \textit{choice functions} in set
theory. Such choice functions associate to each $x \in A$, an element $f(x)$ in $B_x$.
As a fun fact, the Axiom of Choice asserts that this set is nonempty
if every $B_x$ is inhabited.

Another core feature of our language will be type inference.
This will be implemented alongside typechecking, using a fancy
\textit{bidirectional typechecking} algorithm. Don't worry, we'll formalize all
this later. For now, this just means that typechecking and type inference are
mutually recursive processes.

Our language also doesn't have full blown recursion and is
\textit{strongly normalizing} in that every sequence of beta reductions will
always terminate in a unique head normal form. This allows us to safely
normalize all terms to full head normal form.

TODO: explain why our language is strongly normalizing and why this allows us to
normalize everything all the way.

\subsection{Metavariables}
\begin{enumerate}
\item $A$, $B$, $T$, $E$ range over expressions. 
\item $x$, $y$, $z$ range over variables. 
\item $\nu, \, \tau$ range over expressions that are in \textit{full} head normal
  form.
\item $s$ ranges over all sorts, ie $\Type$ and $\Kind$.
\item $n$ ranges over all neutral terms.
  \begin{comment}
    We define this set of expressions inductively via:
    \[ \begin{prooftree} \infer0{\Type} \end{prooftree} \]
    \[ \begin{prooftree} \infer0{x} \end{prooftree} \]
    \[ \begin{prooftree} \hypo{\nu} \infer1{x} \end{prooftree} \]
    \[ \begin{prooftree}
        \hypo{x}
        \hypo{\tau}
        \hypo{\tau'}
        \infer3{\Pi_{x : \tau} \tau'}
      \end{prooftree} \]
    \[ \begin{prooftree} \infer0{\Type} \end{prooftree} \]
  \end{comment}
\end{enumerate}
TODO: Explain head normal form and neutral terms.

We often add primes and subscripts, like $E'$ or $\nu_2$ while referring to these.

\section{Syntax}
Here we describe the concrete ascii syntax for our language.
We have a separate syntax for expressions and statements, the latter of which
function like top-level commands. These will be used by the user to interact
with our language.

\subsection{Syntax for expressions}
\begin{enumerate}
\item \textbf{Sorts} \\
  \[
    \begin{prooftree}
      \infer0{\Type}
    \end{prooftree}
  \]

  \[
    \begin{prooftree}
      \infer0{\Kind}
    \end{prooftree}
  \]

  Following the Calculus of Constructions, we have 2 sorts.

\item \textbf{Variables} \\
  \[
    \begin{prooftree}
      \infer0{x}
    \end{prooftree}
  \]

\item \textbf{Optional parentheses} \\
  \[
    \begin{prooftree}
      \hypo{E}
      \infer1{( \, E \, )}
    \end{prooftree}
  \]
  
\item \textbf{Function abstraction} \\
  \[
    \begin{prooftree}
      \hypo{x}
      \hypo{E}
      \infer2{\fun \, x => E}
    \end{prooftree}
  \]
  \[
    \begin{prooftree}
      \hypo{x_0}
      \hypo{x_1}
      \hypo{\dots}
      \hypo{x_n}
      \hypo{E}
      \infer5{\fun \, x_0 \,\, x_1 \, \dots \, x_n => E}
    \end{prooftree}
  \]
  Note that all functions in our language are unary and so
  $\fun \, x_0 \, x_1 \, \dots \, x_n => E$ is syntactic sugar for
  \[ \fun \, x_0 => (\fun \, x_1 => \dots (\fun x_n => E)) \]

  Simimarly, one can also provide optional type annotations for input variables.
  This is to help the type checker infer the type of a function.
  \[
    \begin{prooftree}
      \hypo{x}
      \hypo{T}
      \hypo{E}
      \infer3{\fun \, (x : T) \, => E}
    \end{prooftree}
  \]

  \[
    \begin{prooftree}
      \hypo{x_i}
      \hypo{T_i}
      \hypo{E}
      \infer3{\fun \, (x_0 : T_0) \,\, (x_1 : T_1) \, \dots \, (x_n : T_n) => E}
    \end{prooftree}
  \]
  We also treat $\fun \, (x_0 : T_0) \,\, (x_1 : T_1) \, \dots \, (x_n : T_n) => E$
  as syntactic sugar for
  \[ \fun \, (x_0 : T_0) => (\fun \, (x_1 : T_1) => \dots (\fun (x_n : T_n) => E)) \]

\item \textbf{Pi and Sigma type} \\
  \[
    \begin{prooftree}
      \hypo{x}
      \hypo{A}
      \hypo{B}
      \infer3{\Pii \, (x : A), \, B}
    \end{prooftree}
  \] 

  \[
    \begin{prooftree}
      \hypo{x_i}
      \hypo{A_i}
      \hypo{B}
      \infer3{\Pii \, (x_0 : A_0) \, (x_1 : A_1) \, \dots \, (x_n : A_n), \, B}
    \end{prooftree}
  \] 
  As with functions, this is syntactic sugar for
  \[ \Pii \, (x_0 : A_0), \, (\Pii \, (x_1 : A_1), \,  \dots \, (\Pii \, (x_n : A_n), \, B)) \]

  The syntax rules for Sigma, ie $\Sigmaa\, (x : A), B$. and syntactic sugar work
  the same as with Pi above. 

  \item \textbf{Sigma constructor} \\
  \[
    \begin{prooftree}
      \hypo{E_1}
      \hypo{E_2}
      \infer2{(E_1, \, E_2)}
    \end{prooftree}  
  \]

  \item \textbf{Sigma eliminators} \\
  \[
    \begin{prooftree}
      \hypo{E}
      \infer1{\fst \, E}
    \end{prooftree}
  \]

  \[
   \begin{prooftree}
    \hypo{E}
    \infer1{\snd \, E}
   \end{prooftree}
 \]

\item \textbf{Type ascriptions} \\
  \[
    \begin{prooftree}
      \hypo{E}
      \hypo{T}
      \infer2{(E : T)}
    \end{prooftree}
  \]
  This functions similarly to other functional languages in that it's used
  mainly to provide a type annotation. For instance, it can be used to help the
  type checker if it's unable to infer the type of an expression.

\item \textbf{Local let bindings} \\
\[
  \begin{prooftree}
    \hypo{x}
    \hypo{E}
    \hypo{E'}
    \infer3{\text{let } x := E \text{ in } E'}
  \end{prooftree}
\]

\item \textbf{Sum type} \\
\[
  \begin{prooftree}
    \hypo{A}
    \hypo{B}
    \infer2{A + B}
  \end{prooftree}
\]

\item \textbf{Sum type constructor} \\
\[
  \begin{prooftree}
    \hypo{E}
    \infer1{\inl E}
  \end{prooftree}
\]

\[
  \begin{prooftree}
    \hypo{E}
    \infer1{\inr E}
  \end{prooftree}
\]

These are meant for introducing the left and right components of a disjoint sum,
ie $\inl E$ constructs an expression of type $A + B$ given an expression $E$ of
type $A$. Similarly, $\inr E$ constructs an $A + B$ given $E$ of type $B$.

\item \textbf{Sum type eliminator} \\
Given expressions $E$ and variables $x$ and $y$, users can perform case analysis
on a sum type via the \verb|match| construct below

\begin{verbatim}
  match E with
  | inl x -> E1
  | inr y -> E2
  end
\end{verbatim}

The ordering of both clauses can be swapped so users can also enter
\begin{verbatim}
  match E with
  | inr y -> E2
  | inl x -> E1
  end
\end{verbatim}

Note here that $x$ is bound in the expression that is $E1$, while $y$ is bound
in $E2$.

This \verb|match| construct is fashioned after the similarly named pattern
matching construct in ML like languages.
However, unlike those, we do not implement actual pattern matching since pattern
matching for dependent types is not an easy problem.
Our \verb|match| construct serves the sole purpose of allowing one to perform
case analysis on sum.

For convenience, we write 
$\match(E, \, x \rightarrow E_1, \, y \rightarrow E_2)$ to denote this
construct.

\end{enumerate}

\subsection{Syntax for statements}
Statements are top level commands through which users interact with our
language. Programs in our language will be \textit{nonempty} sequences of
these statements.

\begin{enumerate}
\item \textbf{Def} \\
\[
  \begin{prooftree}
    \hypo{x}
    \hypo{E}
    \infer2{\deff \, x := E}
  \end{prooftree}
\]
This creates a top level, global definition, binding the variable $x$ to the
expression given by $E$.

\item \textbf{Axiom} \\
\[
  \begin{prooftree}
    \hypo{x}
    \hypo{T}
    \infer2{\axiom \, x : T}
  \end{prooftree}
\]
This defines the variable $x$ to have a type of $T$ with an unknown binding.
The interpreter will treat $x$ like an unknown, indeterminate constant.

\item \textbf{Check} \\
\[
  \begin{prooftree}
    \hypo{E}
    \infer1{\checkk \,\, E}
  \end{prooftree}
\]
This instructs the interpreter to compute the type of the expression $E$ and
output it.

\item \textbf{Eval} \\
\[
  \begin{prooftree}
    \hypo{E}
    \infer1{\evall \,\, E}
  \end{prooftree}
\]
This instructs the interpreter to fully normalize the expression $E$ to full
head normal form.

\end{enumerate}

Note that we often write $\lambda x, \, E$ instead of $\fun \, x => E$ as in the
concrete syntax. Similarly, we also write $\lambda (x : T), \, E$ in place of
$\fun \, (x : T) => x$.
Finally, we use $\Pi_{x : A}B(x)$ to abbreviate $\Pii \, (x : A), \, B$.

\section{Capture avoiding substitutions}
Here we define the notion of capture avoiding substitutions. For this, we
formalize the notion of free variables and substitution.

\subsection{Free variables}
We define the set of free variables of an expression $E$, ie $\FV(E)$ recursively.
\begin{align*}
  \FV (x) &:= \{x\} \\
  \FV (E_1 \, E_2) &:= \FV(E_1) \cup \FV(E_2) \\
  \FV (\lambda \, x, \, E(x)) &:= \FV(E) \setminus \{x\} \\
  \FV (\lambda \, (x : T), \, E(x)) &:= \FV(T) \cup (\FV(E) \setminus \{x\}) \\
  \FV (\text{let } x := E \text{ in } E') &:= \FV((\lambda x, \, E') \,\, E) \\
  \FV (\Pi_{x : A} B(x)) &:= \FV(A) \cup (\FV(B) \setminus \{x\})
\end{align*}
Note that in Pi expressions, the input type $A$ is actually an expression itself
and so may contain free variables. It's important to note that the $\Pi_{x :
  A}$, like a lambda is a binder binding $x$ in $B$. However, this does not bind
$x$ in the input type, $A$. If $x$ does appear in the input type $A$, then it is
free there.

\begin{align*}
 \FV (\Sigma_{x : A} B(x) &:= \FV(\Pi_{x : A} B(x))
\end{align*}

Sigma expressions follow the same rules as with Pi expressions, with $x$ being
bound in the output type $B$ but not in the input type $A$.

\begin{align*}
  \FV (A + B) &:= \FV(A) \cup \FV(B) \\
  \FV (\match(E, x \rightarrow E_1, y \rightarrow E_2)) &:= 
    E \cup (E_1 \setminus \{x\}) \cup (E_2 \setminus \{y\}))
\end{align*}
Note that in match expressions, $x$ is bound in $E_1$ and $y$ is bound in $E_2$.

For the expressions $\fst \, E$, $\snd \, E$, $\inl \, E$ and $\inr \, E$, the
set of free variables is precisely $\FV(E)$, since those keywords are treated
like constants.

\subsection{Substitution}
Here we define $E [x \mapsto E'] := E''$ to mean substituting all free
occurrences of $x$ by $E'$ in the expression $E$ yields another expression $E''$.
\begin{align*}
  x[x \mapsto E] &:= E \\
  y[x \mapsto E] &:= y \quad (\text{if } y \ne x) \\
  (E_1 \,\,\, E_2) [x \mapsto E] &:= (E_1 [x \mapsto E] \,\,\, E_2 [x \mapsto E]) \\
  (\lambda x, \, E) [x \mapsto E'] &:= \lambda x, \, E \\ 
  (\lambda y, \, E) [x \mapsto E'] &:= \lambda y, \, E[x \mapsto E'] \quad (\text{if } y \notin \FV(E')) \\ 
  (\lambda y, \, E) [x \mapsto E'] &:= \lambda z, E[y \mapsto z][x \mapsto E'] \\
                 &\quad \quad \quad (\text{if } z \notin FV(E) \cup FV(E') \cup \{x\}) \\
  (\text{let } y := E \text{ in } E'') [x \mapsto E'] &:= 
    ((\lambda y, \, E) \,\, E'') [x \mapsto E']
\end{align*}

We also define $(\lambda \, (x : T), \, E) [x \mapsto E']$
similar to the case without the optional type annotation above. The only
difference here is we also need to substitute $x$ for $E'$ in the type annotation,
$T$. Note that we do not treat $x$ as being bound in $T$ here.

\begin{align*}
  (\Pi_{x : A} B(x)) [x \mapsto B'] &:= \Pi_{x : A[x \mapsto B']} B(x) \\ 
  (\Pi_{y : A} B(y)) [x \mapsto B'] &:= \Pi_{y : A[x \mapsto B']} B[x \mapsto B'] \quad (\text{if } y \notin \FV(B')) \\ 
  (\Pi_{y : A} B(y)) [x \mapsto B'] &:= \Pi_{z : A[x \mapsto B']} B[y \mapsto z][x \mapsto B'] \\
                 &\quad \quad \quad (\text{if } z \notin FV(B) \cup FV(B') \cup \{x\})
\end{align*}

For Pi expressions, $A$ is an expression and thus when substituting $x$ by
$B'$, we must also perform the substitution in the input type $A$. However, as
$x$ is not bound there, we need not worry about capturing it when substituting there. 

For Sigma expressions, we define $(\Sigma_{x : A} B(x) [x \mapsto B']$ in a
similar fashion as with the case of Pi.

The substitution rules for \verb|fst|, \verb|snd|, \verb|inl| and \verb|inr| are
trivial and thus ommitted.

Finally, we avoid specifying substitution formally for match expressions since it's
tedious. The key thing to note is that in the expression
$\match(E, x \rightarrow E_1, y \rightarrow E_2)$, $x$ is bound in $E_1$ and $y$
is bound in $E_2$.

\section{Big step operational semantics}
\begin{comment}
  https://www.andres-loeh.de/LambdaPi/LambdaPi.pdf
  http://math.andrej.com/2012/11/08/how-to-implement-dependent-type-theory-i/
  
  http://fsl.cs.illinois.edu/images/archive/b/b3/20110221180817!CS522-Spring-2011-PL-book-bigstep.pdf
  https://www.cs.cornell.edu/courses/cs4110/2010fa/lectures/lecture03.pdf
\end{comment}

Here we choose to use a big-step semantics for our language as it fits nicer
with the functional style that we're using to write our recursive interpreter with.
We follow the approach of Plotkin, as described in his
\href{https://web.eecs.umich.edu/~weimerw/2014-6610/reading/plotkin81structural.pdf}
{famous technical report} on structural operational semantics.

\subsection{Expressions}
\subsubsection{Contexts}
Here we take the context, $\Gamma$, to be a list, ie a finite sequence, of triples
of the form
\begin{align*}
  (\text{variable name}, \, \text{type of variable}, \, \text{binding})
\end{align*}
We will use $\emptyset$ to denote the empty context, and $::$ to refer to the
list cons operation.

The binding can be an expression or a special undefined value, which we denote by
\verb|und|.
Contexts have global scope and the top level statements \verb|def|
and \verb|axiom|, return new contexts with updated bindings.
The special \verb|und| value is used for the bindings created by \verb|axiom|
statements and when we want to add a binding for type checking purposes.
In such scenarios, we don't particularly care about the actual binding. We're
only interested in the type of the variable. 

\subsubsection{Overview of judgement forms}
We first give an overview of the main judgment forms which we will define in a
mutually recursive fashion in the subsequent few sections.

\begin{enumerate}
\item $\wf(\Gamma)$ \\
  This asserts that a context $\Gamma$ is well formed.

\item $\Gamma \vdash E \Leftarrow T$ and $\Gamma \vdash E \Rightarrow T$ \\
  These judgments will be used to formalize our bidirectional typechecking
  algorithm. They form the typing rules for our language.
  For now, it suffices to say that $\Gamma \vdash E \Leftarrow T$ formalizes the
  meaning that given an expression $E$ and some type $T$, we may verify that
  $E$ has type $T$ under the context $\Gamma$.

  On the other hand, $\Gamma \vdash E \Rightarrow T$ formalizes the notion of
  \textit{type inference}. It says that from a context $\Gamma$, we may infer
  the type of $E$ to be $T$. 
  
  It should be noted here that this isn't real type
  inference using constraint solving and unification. It's a form of lightweight
  type inference that can infer simple stuff like the return type of a function
  application but not the type parameter in the polymorphic identity function
  $\lambda \, (T : \Type) (x : T), \, x$.

\item $\Gamma \vdash E \Downarrow \nu$ \\
  This $\cdot \vdash \cdot \Downarrow \cdot$ relation is used in our definition of a
  big-step operational semantics to normalize expressions to their full head
  normal form.
  It says that with respect to a context $\Gamma$ containing global bindings, we
  may normalize $E$ to an expression, $\nu$ that is in head normal form.
  % Note that we can do so because our language is a lambda calculus that does
  % not contain full blown recursion and is strongly normalizing. 
\end{enumerate}

\subsubsection{Normalizing expressions}
Before defining our type system given by the 2 judgments,
$\cdot \vdash \cdot \Leftarrow \cdot$ and $\cdot \vdash \cdot \Rightarrow
\cdot$, we must first define $\cdot \vdash \cdot \Downarrow \cdot$, the big
step semantics for evaluating expressions. This is because types are now
expression and we will actually need to perform normalization while
typechecking.

It should be noted at this poinnt that in our semantics to follow, we only normalize
terms after we typecheck them. This means that for types, we always check that
they are well formed before performing any computations on them.

\begin{enumerate}
\item \textbf{Type}
  \[
    \begin{prooftree}
      \hypo{\wf(\Gamma)}
      \infer1{\Gamma \vdash \Type \Downarrow \Type} 
    \end{prooftree}
  \]
\item \textbf{Variables} \\
  \[
    \begin{prooftree}
      \hypo{\wf(\Gamma)}
      \hypo{(x, \, \tau, \, \nu) \in \Gamma}
      \infer2{\Gamma \vdash x \Downarrow \nu}
    \end{prooftree} 
  \]

  \[
    \begin{prooftree}
      \hypo{\wf(\Gamma)}
      \hypo{(x, \, \tau, \, \text{und}) \in \Gamma}
      \infer2{\Gamma \vdash x \Downarrow x}
    \end{prooftree} 
  \]
  
  If there are multiple occurences of the variable $x$ in the context $\Gamma$,
  as is the case when there is variable shadowing, we take the first occurence
  of $x$ in the list that is $\Gamma$.

\item \textbf{Type ascriptions} \\
  \[
    \begin{prooftree}
      \hypo{\Gamma \vdash E \Downarrow \nu}
      \infer1{\Gamma \vdash (E : T) \Downarrow \nu}
    \end{prooftree}
  \]
  This says type ascriptions do not play a role in computation. They're just
  there to help the type checker figure things out and for users to communicate
  their intentions.
 
\item \textbf{Pi elimination, ie function application} \\
  \[
    \begin{prooftree}
      \hypo{\Gamma \vdash E_1 \Downarrow \lambda x, \, \nu_1}
      \hypo{\Gamma \vdash E_2 \Downarrow \nu_2}
      \hypo{\Gamma \vdash \nu_1[x \mapsto \nu_2] \Downarrow \nu}
      \infer3{\Gamma \vdash E_1 \,\, E_2 \Downarrow \nu}
    \end{prooftree}
  \]

  \[
    \begin{prooftree}
      \hypo{\Gamma \vdash E_1 \Downarrow n}
      \hypo{\Gamma \vdash E_2 \Downarrow \nu}
      \infer2{\Gamma \vdash E_1 \,\, E_2 \Downarrow n \,\, \nu}
    \end{prooftree}
  \]

  The last rule handle the cases of neutral expressions.

\item \textbf{Normalizing under lambdas} \\
  \[
    \begin{prooftree}
      \hypo{\Gamma \vdash E \Downarrow \nu}
      \infer1{\Gamma \vdash \lambda x, \, E, \, \Downarrow \lambda x, \, \nu}
    \end{prooftree}
  \]
  \[
    \begin{prooftree}
      \hypo{\Gamma \vdash \lambda x, \, E \Downarrow \nu}
      \infer1{\Gamma \vdash \lambda \, (x : T), \, E, \, \Downarrow \nu}
    \end{prooftree}
  \]
  This second rule says that optional type ascriptions do not play a role in
  normalization, ie we just ignore them.

\item \textbf{Pi} \\
  \[
    \begin{prooftree}
      \hypo{\Gamma \vdash A \Downarrow \tau}
      \hypo{\Gamma \vdash B(x) \Downarrow \tau'(x)}
      \infer2{\Gamma \vdash \Pi_{x : A} B(x) \Downarrow \Pi_{x : \tau} \tau'(x)}
    \end{prooftree}
  \]

\item \textbf{Local let binding} \\
    \[
      \begin{prooftree}
        \hypo{\Gamma \vdash E \Downarrow \nu}
        \hypo{\Gamma \vdash E'[x \mapsto \nu] \Downarrow \nu'}
        \infer2{\Gamma \vdash \text{let } x := E \text{ in } E' \Downarrow \nu'}
      \end{prooftree}
    \]

\item \textbf{Normalizing under pair constructor} \\
    \[
      \begin{prooftree}
        \hypo{\Gamma \vdash E_1 \Downarrow \nu_1}
        \hypo{\Gamma \vdash E_2 \Downarrow \nu_2}
        \infer2{\Gamma \vdash \langle E_1, \, E_2 \rangle \Downarrow \langle
          \nu_1, \, \nu_2 \rangle}
      \end{prooftree}
    \]

  \item \textbf{Sigma} \\
    \[
      \begin{prooftree}
        \hypo{\Gamma \vdash A \Downarrow \tau}
        \hypo{\Gamma \vdash B \Downarrow \tau'(x)}
        \infer2{\Gamma \vdash \Sigma_{x : A}B(x) \Downarrow \Sigma_{x : \tau}\tau'(x)}
      \end{prooftree}
    \]

  \item \textbf{Sigma elimination} \\
   \[
    \begin{prooftree}
      \hypo{\Gamma \vdash E \Downarrow (\nu_1, \, \nu_2)}
      \infer1{\Gamma \vdash \fst E \Downarrow \nu_1}
    \end{prooftree}
  \]

  \[
   \begin{prooftree}
    \hypo{\Gamma \vdash E \Downarrow n}
     \infer1{\Gamma \vdash \fst E \Downarrow \fst n}
   \end{prooftree}
  \]

  \[
    \begin{prooftree}
      \hypo{\Gamma \vdash E \Downarrow (\nu_1, \, \nu_2)}
      \infer1{\Gamma \vdash \snd E \Downarrow \nu_2}
    \end{prooftree}
  \]

  \[
   \begin{prooftree}
     \hypo{\Gamma \vdash E \Downarrow n}
     \infer1{\Gamma \vdash \snd E \Downarrow \snd n}
   \end{prooftree}
  \]

  % \[
  %   \begin{prooftree}
  %     \hypo{\Gamma \vdash E \Downarrow \langle \nu_1, \, \nu_2 \rangle}
  %     \hypo{\Gamma \vdash E'[x_1 \mapsto \nu_1][x_2 \mapsto \nu_2] \Downarrow \nu}
  %     \infer2{\Gamma \vdash \text{let } (x_1, \, x_2) := E \text{ in } E' \Downarrow \nu}
  %   \end{prooftree}
  % \]

  \item \textbf{Normalizing under sum data constructors}
  \[
    \begin{prooftree}
      \hypo{\Gamma \vdash E \Downarrow \nu}
      \infer1{\Gamma \vdash \inl \, E \Downarrow \inl \, \nu}
    \end{prooftree}  
  \]

  \[
    \begin{prooftree}
     \hypo{\Gamma \vdash E \Downarrow \inl \nu}
     \infer1{\Gamma \vdash \inr \, E \Downarrow \nu}
    \end{prooftree}  
  \]

  \item \textbf{Normalizing sum eliminator}
  \[
    \begin{prooftree}
      \hypo{\Gamma \vdash E \Downarrow \inl \nu}
      \hypo{\Gamma \vdash E_1[x \mapsto \nu] \Downarrow \nu_1}
      \infer2{\Gamma \vdash 
        \match(E, \, x \rightarrow E_1, \, y \rightarrow E_2)
        \Downarrow \nu_1}
    \end{prooftree}  
  \]

  \[
   \begin{prooftree}
     \hypo{\Gamma \vdash E \Downarrow \inr \nu}
     \hypo{\Gamma \vdash E_2[y \mapsto \nu] \Downarrow \nu_2}
     \infer2{\Gamma \vdash 
       \match(E, \, x \rightarrow E_1, \, y \rightarrow E_2)
       \Downarrow E_2[y \mapsto \nu_2]}
   \end{prooftree}  
  \]

  \[
   \begin{prooftree}
    \hypo{\Gamma \vdash E \Downarrow n}
      \hypo{\Gamma \vdash E_1 \Downarrow \nu_1}
      \hypo{\Gamma \vdash E_2 \Downarrow \nu_2}
      \infer3{\Gamma \vdash \match(E, \, x \rightarrow E_1, \, y \rightarrow E_2) 
               \Downarrow
               \match(n, \, x \rightarrow \nu_1, \, y \rightarrow \nu_2) 
               }
   \end{prooftree}  
  \]

\end{enumerate}

\subsubsection{Well formed context and typing judgments}
\begin{definition} [Well formed type and type constructor]
  An expression $T$ is said to be a well-formed type with respect to the context
  $\Gamma$ if it satisfies
  \[ \Gamma \vdash T \Leftarrow s \]

  Informally, we can think of $\Type$ as the ``type of all small types'' and so
  all small well formed types are those expressions satisfying
  $\Gamma \vdash T \Leftarrow \Type$, ie they can be checked to have type $\Type$.

  Those $T$ satisfying $\Gamma \vdash T \Leftarrow \Kind$ instead represent the
  type of type constructors. In other words, the type of a type constructor is a
  Kind, just like in Haskell. 
\end{definition}

\begin{definition} [Well formed context]
  Note that our definition below implies that expressions and types that we
  store in our context are fully normalized to head normal form.

  \begin{enumerate}
  \item \textbf{Base case} \\
    \[
      \begin{prooftree}
        \hypo{\wf(\emptyset)}
      \end{prooftree}
    \]

  \item \textbf{Inductive cases} \\
    \[
      \begin{prooftree}
        \hypo{\wf(\Gamma)}
        \hypo{\Gamma \vdash \tau \Rightarrow s}
        \infer2{\wf ((x, \, \tau, \, \nu) :: \Gamma)}
      \end{prooftree}
    \]

    \[
      \begin{prooftree}
        \hypo{\wf(\Gamma)}
        \hypo{\Gamma \vdash \tau \Rightarrow s}
        \infer2{\wf ((x, \, \tau, \, \text{und}) :: \Gamma)}
      \end{prooftree}
    \]
  \end{enumerate}
\end{definition}

\begin{definition} [Bidrectional typechecking]
  Here we define the 2 \textit{mutually recursive} relations
  \begin{enumerate}
  \item$\cdot \vdash \cdot \Rightarrow \cdot $ which corresponds to \textit{inference}
  \item$\cdot \vdash \cdot \Leftarrow \cdot$ which corresponds to \textit{checking}
  \end{enumerate}

  The idea is that there are some expressions for which it is easier to
  \textit{infer},
  ie compute the type directly, while for others, it is easier to have the user
  supply a type annotation and then \textit{check} that it is correct.

  As a rule of thumb, it is often easier to check the type for term
  introduction rules while for elimination rules, it is usually easier to infer
  the type.

  \begin{enumerate}

  \item \textbf{Type ascriptions} \\
    \[
      \begin{prooftree}
        \hypo{\Gamma \vdash T \Rightarrow s}
        \hypo{\Gamma \vdash T \Downarrow \tau}
        \hypo{\Gamma \vdash E \Leftarrow \tau}
        \infer3{\Gamma \vdash (E : T) \Rightarrow \tau}
      \end{prooftree}
    \]
    Optional type ascriptions allow the interpreter to infer the type of an
    expression. This is useful for lambda abstractions in particular because it's
    kinda hard to infer the type of a function like $\lambda x, \, x$ without any
    further contextual information.

    \[
      \begin{prooftree}
        \hypo{\Gamma \vdash E \Leftarrow \Kind}
        \infer1{\Gamma \vdash (E : \Kind) \Rightarrow \Kind}
      \end{prooftree}
    \]
    Note that the first rule doesn't allow users to assert that $(E : \Kind)$
    since there is no $s$ with $\Gamma \vdash \Kind \Rightarrow s$.
    This second rule allows users to assert that $\Type$ and type constructors
    have type $\Kind$.
    
  \item \textbf{Var} \\
    \[
      \begin{prooftree}
        \hypo{\wf(\Gamma)}
        \hypo{(x, \, \tau, \, v) \in \Gamma}
        \infer2{\Gamma \vdash x \Rightarrow \tau}
      \end{prooftree}
    \]
    This says that we may infer the type of a variable if the type information
    is already in our context.
    Note that in this rule, we also allow $\nu$ to be \verb|und|.

  \item \textbf{Type} \\
    \[
      \begin{prooftree}
        \hypo{\wf(\Gamma)}
        \infer1{\Gamma \vdash \Type \Rightarrow \Kind}
      \end{prooftree}
    \]

    % \[
    %   \begin{prooftree}
    %     \hypo{\wf(\Gamma)}
    %     \infer1{\Gamma \vdash \Type \Rightarrow \Type}
    %   \end{prooftree}
    % \]

  \item \textbf{Check} \\
    \[
      \begin{prooftree}
        \hypo{\Gamma \vdash E \Rightarrow \tau'}
        \hypo{\tau \equiv_{\alpha \beta} \tau'}
        \infer2{\Gamma \vdash E \Leftarrow \tau}
      \end{prooftree}
    \]
    This says that to check if $E$ has type $\tau$ with respect to a context
    $\Gamma$, we may first infer the type of $E$. Suppose it is $\tau'$. Then
    if we also find that $\tau$ and $\tau'$ are $\alpha$ and $\beta$ equivalent
    to each other, we may conclude that $E$ indeed has type $\tau$.

  \item \textbf{Pi formation} \\
    \[
      \begin{prooftree}
        \hypo{\Gamma \vdash A \Rightarrow s_1}
        \hypo{\Gamma \vdash A \Downarrow \tau}
        \hypo{(x, \, \tau, \, \text{und}) :: \Gamma \vdash B(x) \Rightarrow s_2}
        \infer3{\Gamma \vdash \Pi_{x : A}B(x) \Rightarrow s_2}
      \end{prooftree}
    \]
    Note that this is a rule schema with the metavariables 
    $s_1, \, s_2 \in \{\Type, \, \Kind\}$

    % \[
    %   \begin{prooftree}
    %     \hypo{\Gamma \vdash A \Leftarrow \Type}
    %     \hypo{\Gamma \vdash A \Downarrow \tau}
    %     \hypo{(x, \, \tau, \, \text{undefined}) :: \Gamma \vdash B(x) \Leftarrow \Type}
    %     \infer3{\Gamma \vdash \Pi_{x : A}B(x) \Rightarrow \Type}
    %   \end{prooftree}
    % \]

  \item \textbf{Pi introduction} \\
    \[
      \begin{prooftree}
        \hypo{(x, \, \tau, \, \text{und}) :: \Gamma \vdash E \Leftarrow \tau'(x)}
        \infer1{\Gamma \vdash \lambda x, \, E \Leftarrow \Pi_{x : \tau}\tau'(x)}
      \end{prooftree}
    \]
    Note that in the event that the input variable of the lambda abstraction and
    Pi are different, then we must perform an $\alpha$ renaming so that the
    variable being bound in $(\lambda x, \, E)$ and $\Pi_{y : \tau} \tau'(y)$
    are the same.

    \[
      \begin{prooftree}
        \hypo{\Gamma \vdash T \Rightarrow s}
        \hypo{\Gamma \vdash T \Downarrow \tau}
        \hypo{(x, \, \tau, \, \text{und}) :: \Gamma \vdash E \Rightarrow \tau'(x)}
        \infer3{\Gamma \vdash \lambda \, (x : T), \, E \Rightarrow \Pi_{x : \tau}\tau'(x)}
      \end{prooftree}
    \]
    The second rule says that if the user type annotates the input argument of
    the function, then we can try to infer the type of the output and
    consequently, the type of the function as a whole.

  \item \textbf{Function application, ie Pi elimination} \\
    \[
      \begin{prooftree}
        \hypo{\Gamma \vdash E_1 \Rightarrow \Pi_{x : \tau}\tau'(x)}
        \hypo{\Gamma \vdash E_2 \Leftarrow \tau}
        \hypo{\Gamma \vdash \tau'[x \mapsto E_2] \Downarrow \tau''}
        \infer3{E_1 \,\, E_2 \Rightarrow \tau''}
      \end{prooftree}
    \]

  \item \textbf{Local let binding} \\
    \[
      \begin{prooftree}
        \hypo{\Gamma \vdash E \Rightarrow \tau}
        \hypo{(x, \, \tau, \, \text{und}) :: \Gamma \vdash E' \Rightarrow \tau'(x)}
        \hypo{\Gamma \vdash \tau'[x \mapsto E] \Downarrow \tau''}
        \infer3{\Gamma \vdash \text{let } x := E \text{ in } E' \Rightarrow \tau''}
      \end{prooftree}
    \]

  \item \textbf{Sigma formation} \\
    \[
      \begin{prooftree}
        \hypo{\Gamma \vdash A \Rightarrow s_1}
        \hypo{\Gamma \vdash A \Downarrow \tau}
        \hypo{(x, \, \tau, \, \text{und}) :: \Gamma \vdash B(x) \Rightarrow s_2}
        \infer3{\Gamma \vdash \Sigma_{x : A}B(x) \Rightarrow s_2}
      \end{prooftree}
    \]

  \item \textbf{Sigma introduction} \\
    \[
      \begin{prooftree}
        \hypo{\Gamma \vdash E_1 \Leftarrow \tau_1}
        \hypo{\Gamma \vdash \tau_2[x \mapsto E_1] \Downarrow \tau_2'}
        \hypo{\Gamma \vdash E_2 \Leftarrow \tau_2'}
        \infer3{\Gamma \vdash \langle  E_1, \, E_2 \rangle \Leftarrow 
          \Sigma_{x : \tau_1}\tau_2(x)}
      \end{prooftree}
    \]
    Here we don't have a type inference rule because doing so would require us
    to infer that $E_2$ has type $\tau_2[x \mapsto E_1]$ and then undo the
    substitution to get $\tau_2(x)$.

  \item \textbf{Sigma elimination} \\
    \[
      \begin{prooftree}
        \hypo{\Gamma \vdash E \Rightarrow \Sigma_{x : \tau_1}\tau_2(x)}
        \infer1{\Gamma \vdash \fst E \Rightarrow \tau_1}
      \end{prooftree}
    \]

    \[
      \begin{prooftree}
        \hypo{\Gamma \vdash E \Rightarrow \Sigma_{x : \tau_1}\tau_2(x)}
        \hypo{\Gamma \vdash \tau_2[x \mapsto \fst E] \Downarrow \tau_2'}
        \infer2{\Gamma \vdash \snd E \Rightarrow \tau_2'}
      \end{prooftree}
    \]

    % \[
    %   \begin{prooftree}
    %     \hypo{E \Rightarrow \Sigma_{x : \tau_1}\tau_2(x)}
    %     \hypo{\Gamma \vdash \tau_2[x \mapsto \fst E] \Downarrow \tau_2'}
    %     \hypo{(y, \, \tau_2', \, \text{und}) :: (x, \, \tau_1, \, \text{und}) :: \Gamma
    %           \vdash E' \rightarrow \tau}
    %     \infer3{\Gamma \vdash \text{let } (x, \, y) := E \text{ in } E' \Rightarrow \tau}
    %   \end{prooftree} 
    % \]

    % \item \textbf{Pair destructuring} \\
    % \[
    %   \begin{prooftree}
    %     \hypo{\Gamma \vdash E \Rightarrow \Sigma_{x : \tau_1}\tau_2(x)}
    %     % \hypo{(x_2, \, \tau_2[x \mapsto x_1], \text{und}) :: 
    %     %   (x_1, \, \tau_1, \, \text{und}) :: 
    %     %   \Gamma \vdash E' \Rightarrow \tau(x_1, \, x_2)}
    %     \hypo{\Gamma \vdash E' [x_1 \mapsto \fst E][x_2 \mapsto \snd E]
    %       \Rightarrow \tau}
    %     \infer2{\Gamma \vdash (\text{let } (x_1, \, x_2) := E
    %       \text{ in } E') \Rightarrow \tau}
    %   \end{prooftree}
    % \]

    \item \textbf{Sum formation} \\
    \[
      \begin{prooftree}
       \hypo{\Gamma \vdash A \Leftarrow \Type} 
       \hypo{\Gamma \vdash B \Leftarrow \Type} 
       \infer2{\Gamma \vdash A + B \Rightarrow \Type} 
      \end{prooftree}
    \]
    Note that this rule says that users can only make construct a coproduct
    out of types that live in the universe $\Type$, not $\Kind$. 

    \item \textbf{Sum introduction} \\
    \[
      \begin{prooftree}
       \hypo{\Gamma \vdash E \Leftarrow \tau_1} 
       \infer1{\Gamma \vdash \inl E \Leftarrow \tau_1 + \tau_2} 
      \end{prooftree}
    \]

    \[
     \begin{prooftree}
      \hypo{\Gamma \vdash E \Leftarrow \tau_2} 
      \infer1{\Gamma \vdash \inr E \Leftarrow \tau_1 + \tau_2} 
     \end{prooftree}
   \]

   \item \textbf{Sum elimination} \\
    \[
    \begin{prooftree}
      \hypo{\Gamma \vdash E \Rightarrow \tau_1 + \tau_2}
      \hypo{(x, \, \tau_1, \, \text{und}) :: \Gamma \vdash E_1 \Rightarrow \tau_1'}
      \hypo{(y, \, \tau_2, \, \text{und}) :: \Gamma \vdash E_2 \Rightarrow \tau_2'}
      \hypo{\tau_1' \equiv_{\alpha \beta} \tau_2'}
      \infer4{\Gamma \vdash \match(E, \, x \rightarrow E_1, \, y \rightarrow E_2)
               \Rightarrow \tau_1'}
    \end{prooftree}  
  \]

  % \item \textbf{Local let binding} \\
  %   \[
  %     \begin{prooftree}
  %       \hypo{\Gamma \vdash E \Rightarrow \tau}
  %       \hypo{(x, \, \tau, \, \text{undefined}) :: \Gamma \vdash E' \Rightarrow \tau'(x)}
  %       \hypo{\Gamma \vdash \tau'[x \mapsto E] \Downarrow \tau''}
  %       \infer3{\Gamma \vdash (\text{let } x := E \text{ in } E') \Rightarrow \tau''}
  %     \end{prooftree}
  %   \]

  %   \[
  %     \begin{prooftree}
  %       \hypo{\Gamma \vdash (\text{let } x := (E : T)) \text{ in } E' \Rightarrow \tau}
  %       \infer1{\Gamma \vdash (\text{let } (x : T) := E \text{ in } E') \Rightarrow \tau}
  %     \end{prooftree}
  %   \]
  %   This second rule handles the case when an optional type annotation is given
  %   for the variable $x$.

  % \item \textbf{Pair destructuring} \\
  %   \[
  %     \begin{prooftree}
  %       % \hypo{\Gamma \vdash E \Rightarrow \Sigma_{x : \tau_1}\tau_2(x)}
  %       \hypo{\Gamma \vdash (\lambda z, \, E'[x_1 \mapsto \fst z][x_2 \mapsto
  %         \snd z]) \,\, E
  %         \Rightarrow \tau}
  %         \hypo{z \notin \FV(E')}
  %         \infer2{\Gamma \vdash (\text{let } \langle x_1, \, x_2 \rangle := E
  %         \text{ in } E') \Rightarrow \tau}
  %       \end{prooftree}
  %     \]

  \end{enumerate}
\end{definition}

The nice thing about these rules is that we can translate it almost directly
into a typechecking and inference algorithm!
More precisely, we may translate $\Gamma \vdash E \Rightarrow T$ into a
function called $\text{check}(\Gamma, \, E, \, T)$ which checks if the
expression $E$ really has the type $T$ given a context $\Gamma$.

Similarly, $\Gamma \vdash E \Leftarrow T$ gives us a the function
$\text{infer}(\Gamma, \, E)$ which outputs the inferred type of $E$ given
the context $\Gamma$.

In the literature, such rules are called \textit{syntax directed}, as the
algorithm closely follows the formalization of the corresponding judgments.

It's worth noting that $\eta$ equivalence is not respected by our type system.
By that we mean that the following rule does not hold:
\[
  \begin{prooftree}
    \hypo{\Gamma \vdash E \Rightarrow \tau'}
    \hypo{\tau \equiv_{\eta} \tau'}
    \infer2{\Gamma \vdash E \Leftarrow \tau}
 \end{prooftree}
\]
In other words, 2 types that are $\eta$ equivalent to one another will 
\textit{not} be judged as equal types.

One way to fix this is to allow the type checker to eta expand terms, via
eliminating and then applying the data constructor, after computing the beta
normal form. However, for simplicity, we chose to follow the original 
formulation of the calculus of constructions and ignore this.

% Another kind of equivalence is \textit{eta equivalence}. To explain this,
% consider the terms $\lambda x, \, B \, x$ and $B$. Both of them have exactly the
% same behavior. This is one form of eta equivalence.
% In lambda calculus terms, the latter is obtained from the former by an eta
% reduction.

% Now, we may view the lambda as the data constructor for
% the Pi type and function application as the eliminator.
% Then the expression $\lambda x, \, B \, x$ is obtained from $B$ by first
% eliminating $B$ to obtain $B x$ and then using the lambda
% constructor to obtain $\lambda x, \, B \, x$.

% More generally, we say that 2 terms are eta equivalent to each other if one can
% be obtained from the other by eliminating the term and then reconstructing the
% original term using the constructor. For instance, if we had a binary coproduct,
% we could project out both components and then recombine them to obtain the
% original sum.

% Unfortunately our type system is intensional rather than extensional in
% that the following rule does \textit{not} hold: 
% \[
%   \begin{prooftree}
%     \hypo{\Gamma \vdash E \Leftarrow \tau}
%     \hypo{\tau =_\eta \tau'}
%     \infer2{\Gamma \vdash E \Leftarrow \tau'}
%   \end{prooftree}
% \]

\subsection{Programs}
\begin{comment}
  Follows this approach:
  https://www.cs.cornell.edu/courses/cs6110/2013sp/lectures/lec05-sp13.pdf
\end{comment}

Programs in our language are nonempty sequences of statements, with each
statement being built from some expression.
Following \href{https://web.eecs.umich.edu/~weimerw/2014-6610/reading/plotkin81structural.pdf}
{Plotkin's approach} to structural operational semantics, we formalize
the execution of program by defining a \textit{transition system}.

Such transition systems behave like finite automata, although they aren't
restricted to having finitely many states or transitions. In the next section,
we define the notion of a configuration, which plays the same role
as states in automata. Thereafter, we formalize the big-step transition relation
for executing programs.

\subsubsection{Configuration/state of program}
A configuration has the form
\begin{align*}
  (\Gamma, \, \langle S_0; \, \dots; \, S_n \rangle, \, E)
\end{align*}
Configurations are triples representing the instantaneous state during an
execution of a program. Here, $\Gamma$ denotes the current state of the
global context and the sequence $\langle S_0; \, \dots; \, S_n \rangle$ denotes
the next statements to be executed. 
The expression $E$ is used to indicate the output of the previously executed
statement.

With this view, given a program, $\langle S_0; \, \dots; \, S_n
\rangle$, we also define:
\begin{enumerate}
\item \textbf{Initial configuration} \\
  $ (\emptyset, \, \langle S_0; \, \dots; \, S_n \rangle, \, \Type)$

  Initially, our global context is empty. Also, \verb|Type| is merely a dummy
  value and could have very well be replaced by \verb|Kind|.

\item \textbf{Final configurations} \\
  These are all configurations of the form
  $(\Gamma, \, \langle \rangle, \, E)$
\end{enumerate}

In the next section, we define the big-step transition relation for programs $\cdot
\Downarrow \cdot$ using
this notion of a configuration.

\subsubsection{Big step semantics for programs}
We define the big step transition relation, $\cdot \Downarrow \cdot$, via a
new judgment form in our calculus.

It should be noted that in the rules below, we interleave type checking and 
evaluation, ie we type check and then evaluate each statement, one at a time,
carrying the context along as we go.
We do not statically type check the whole program first, then evaluate
afterwards.

\begin{enumerate}
\item \textbf{Check} \\
  \[
    \begin{prooftree}
      \hypo{\Gamma \vdash E \Rightarrow \tau}
      \infer1{(\Gamma, \, \langle \checkk \, E \rangle, \, E') \Downarrow
        (\Gamma, \, \langle \rangle, \, \tau)}
    \end{prooftree}
  \]

\item \textbf{Eval} \\
  \[
    \begin{prooftree}
      \hypo{\Gamma \vdash E \Rightarrow \tau}
      \hypo{\Gamma \vdash E \Downarrow \nu}
      \infer2{(\Gamma, \, \langle \evall \, E \rangle, \, E') \Downarrow
        (\Gamma, \, \langle \rangle, \, \nu)}
    \end{prooftree}
  \]

\item \textbf{Axiom} \\
  \[
    \begin{prooftree}
      \hypo{\Gamma \vdash T \Rightarrow s}
      \hypo{\Gamma \vdash T \Downarrow \tau}
      \infer2{(\Gamma, \, \langle \axiom x : T \rangle, \, E) \Downarrow
        ((x, \, \tau, \, \text{und}) :: \Gamma, \, \langle \rangle, \, x)}
    \end{prooftree}
  \]

\item \textbf{Def} \\
  \[
    \begin{prooftree}
      \hypo{\Gamma \vdash E \Rightarrow \tau}
      \hypo{\Gamma \vdash E \Downarrow \nu}
      \infer2{(\Gamma, \, \langle \deff x := E \rangle, \, E') \Downarrow
        ((x, \, \tau, \, \nu) :: \Gamma, \, \langle \rangle, \, x)}
    \end{prooftree}
  \]

\item \textbf{Sequences of statements} \\
  \[
    \begin{prooftree}
      \hypo{(\Gamma, \, \langle S_0 \rangle, \, E) \Downarrow
        (\Gamma', \, \langle \rangle, \, E') }
      \hypo{(\Gamma', \, \langle S_1; \, \dots; \, S_n \rangle, \, E) \Downarrow
        (\Gamma'', \, \langle \rangle, \, E'') }
      \infer2{(\Gamma, \, \langle S_0; \, S_1; \, \dots; \, S_n \rangle, \, E) \Downarrow
        (\Gamma'', \, \langle \rangle, \, E'') }
    \end{prooftree}
  \]

\end{enumerate}

\subsubsection{Putting the transition system together}
Letting $S$ denote the (infinite) set of all configurations, and defining
\[ F := \{ (\Gamma, \, \langle \rangle, \, E) \in S \, | \, E \text{ expression} \} \]

we obtain a transition system, given by the tuple
\begin{align*}
  \langle S, \, \Downarrow, \, 
  (\emptyset, \, \langle S_0; \, \dots; \, S_n \rangle, \, \Type), \, F \rangle
\end{align*}

Notice how our formalization mimics the definition of a finite state automaton.

\subsubsection{Output expression of evaluating a program}
With these rules, given a user-entered program, say $\langle S_0;
\dots; S_n \rangle$, we define the
output expression of a program to be the $E$ such that
\begin{align*}
  (\emptyset, \, \langle S_0; \, \dots \; S_n \rangle, \, \Type) \Downarrow
  (\Gamma, \, \langle \rangle, \, E)
\end{align*}

In other words, the expression output to the user is the expression obtained by
beginning with the initial configuration and then recursively evaluating
until we reach a final configuration.
\end{document}